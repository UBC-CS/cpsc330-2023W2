\documentclass[11pt]{article}

    \usepackage[breakable]{tcolorbox}
    \usepackage{parskip} % Stop auto-indenting (to mimic markdown behaviour)
    

    % Basic figure setup, for now with no caption control since it's done
    % automatically by Pandoc (which extracts ![](path) syntax from Markdown).
    \usepackage{graphicx}
    % Maintain compatibility with old templates. Remove in nbconvert 6.0
    \let\Oldincludegraphics\includegraphics
    % Ensure that by default, figures have no caption (until we provide a
    % proper Figure object with a Caption API and a way to capture that
    % in the conversion process - todo).
    \usepackage{caption}
    \DeclareCaptionFormat{nocaption}{}
    \captionsetup{format=nocaption,aboveskip=0pt,belowskip=0pt}

    \usepackage{float}
    \floatplacement{figure}{H} % forces figures to be placed at the correct location
    \usepackage{xcolor} % Allow colors to be defined
    \usepackage{enumerate} % Needed for markdown enumerations to work
    \usepackage{geometry} % Used to adjust the document margins
    \usepackage{amsmath} % Equations
    \usepackage{amssymb} % Equations
    \usepackage{textcomp} % defines textquotesingle
    % Hack from http://tex.stackexchange.com/a/47451/13684:
    \AtBeginDocument{%
        \def\PYZsq{\textquotesingle}% Upright quotes in Pygmentized code
    }
    \usepackage{upquote} % Upright quotes for verbatim code
    \usepackage{eurosym} % defines \euro

    \usepackage{iftex}
    \ifPDFTeX
        \usepackage[T1]{fontenc}
        \IfFileExists{alphabeta.sty}{
              \usepackage{alphabeta}
          }{
              \usepackage[mathletters]{ucs}
              \usepackage[utf8x]{inputenc}
          }
    \else
        \usepackage{fontspec}
        \usepackage{unicode-math}
    \fi

    \usepackage{fancyvrb} % verbatim replacement that allows latex
    \usepackage{grffile} % extends the file name processing of package graphics
                         % to support a larger range
    \makeatletter % fix for old versions of grffile with XeLaTeX
    \@ifpackagelater{grffile}{2019/11/01}
    {
      % Do nothing on new versions
    }
    {
      \def\Gread@@xetex#1{%
        \IfFileExists{"\Gin@base".bb}%
        {\Gread@eps{\Gin@base.bb}}%
        {\Gread@@xetex@aux#1}%
      }
    }
    \makeatother
    \usepackage[Export]{adjustbox} % Used to constrain images to a maximum size
    \adjustboxset{max size={0.9\linewidth}{0.9\paperheight}}

    % The hyperref package gives us a pdf with properly built
    % internal navigation ('pdf bookmarks' for the table of contents,
    % internal cross-reference links, web links for URLs, etc.)
    \usepackage{hyperref}
    % The default LaTeX title has an obnoxious amount of whitespace. By default,
    % titling removes some of it. It also provides customization options.
    \usepackage{titling}
    \usepackage{longtable} % longtable support required by pandoc >1.10
    \usepackage{booktabs}  % table support for pandoc > 1.12.2
    \usepackage{array}     % table support for pandoc >= 2.11.3
    \usepackage{calc}      % table minipage width calculation for pandoc >= 2.11.1
    \usepackage[inline]{enumitem} % IRkernel/repr support (it uses the enumerate* environment)
    \usepackage[normalem]{ulem} % ulem is needed to support strikethroughs (\sout)
                                % normalem makes italics be italics, not underlines
    \usepackage{soul}      % strikethrough (\st) support for pandoc >= 3.0.0
    \usepackage{mathrsfs}
    

    
    % Colors for the hyperref package
    \definecolor{urlcolor}{rgb}{0,.145,.698}
    \definecolor{linkcolor}{rgb}{.71,0.21,0.01}
    \definecolor{citecolor}{rgb}{.12,.54,.11}

    % ANSI colors
    \definecolor{ansi-black}{HTML}{3E424D}
    \definecolor{ansi-black-intense}{HTML}{282C36}
    \definecolor{ansi-red}{HTML}{E75C58}
    \definecolor{ansi-red-intense}{HTML}{B22B31}
    \definecolor{ansi-green}{HTML}{00A250}
    \definecolor{ansi-green-intense}{HTML}{007427}
    \definecolor{ansi-yellow}{HTML}{DDB62B}
    \definecolor{ansi-yellow-intense}{HTML}{B27D12}
    \definecolor{ansi-blue}{HTML}{208FFB}
    \definecolor{ansi-blue-intense}{HTML}{0065CA}
    \definecolor{ansi-magenta}{HTML}{D160C4}
    \definecolor{ansi-magenta-intense}{HTML}{A03196}
    \definecolor{ansi-cyan}{HTML}{60C6C8}
    \definecolor{ansi-cyan-intense}{HTML}{258F8F}
    \definecolor{ansi-white}{HTML}{C5C1B4}
    \definecolor{ansi-white-intense}{HTML}{A1A6B2}
    \definecolor{ansi-default-inverse-fg}{HTML}{FFFFFF}
    \definecolor{ansi-default-inverse-bg}{HTML}{000000}

    % common color for the border for error outputs.
    \definecolor{outerrorbackground}{HTML}{FFDFDF}

    % commands and environments needed by pandoc snippets
    % extracted from the output of `pandoc -s`
    \providecommand{\tightlist}{%
      \setlength{\itemsep}{0pt}\setlength{\parskip}{0pt}}
    \DefineVerbatimEnvironment{Highlighting}{Verbatim}{commandchars=\\\{\}}
    % Add ',fontsize=\small' for more characters per line
    \newenvironment{Shaded}{}{}
    \newcommand{\KeywordTok}[1]{\textcolor[rgb]{0.00,0.44,0.13}{\textbf{{#1}}}}
    \newcommand{\DataTypeTok}[1]{\textcolor[rgb]{0.56,0.13,0.00}{{#1}}}
    \newcommand{\DecValTok}[1]{\textcolor[rgb]{0.25,0.63,0.44}{{#1}}}
    \newcommand{\BaseNTok}[1]{\textcolor[rgb]{0.25,0.63,0.44}{{#1}}}
    \newcommand{\FloatTok}[1]{\textcolor[rgb]{0.25,0.63,0.44}{{#1}}}
    \newcommand{\CharTok}[1]{\textcolor[rgb]{0.25,0.44,0.63}{{#1}}}
    \newcommand{\StringTok}[1]{\textcolor[rgb]{0.25,0.44,0.63}{{#1}}}
    \newcommand{\CommentTok}[1]{\textcolor[rgb]{0.38,0.63,0.69}{\textit{{#1}}}}
    \newcommand{\OtherTok}[1]{\textcolor[rgb]{0.00,0.44,0.13}{{#1}}}
    \newcommand{\AlertTok}[1]{\textcolor[rgb]{1.00,0.00,0.00}{\textbf{{#1}}}}
    \newcommand{\FunctionTok}[1]{\textcolor[rgb]{0.02,0.16,0.49}{{#1}}}
    \newcommand{\RegionMarkerTok}[1]{{#1}}
    \newcommand{\ErrorTok}[1]{\textcolor[rgb]{1.00,0.00,0.00}{\textbf{{#1}}}}
    \newcommand{\NormalTok}[1]{{#1}}

    % Additional commands for more recent versions of Pandoc
    \newcommand{\ConstantTok}[1]{\textcolor[rgb]{0.53,0.00,0.00}{{#1}}}
    \newcommand{\SpecialCharTok}[1]{\textcolor[rgb]{0.25,0.44,0.63}{{#1}}}
    \newcommand{\VerbatimStringTok}[1]{\textcolor[rgb]{0.25,0.44,0.63}{{#1}}}
    \newcommand{\SpecialStringTok}[1]{\textcolor[rgb]{0.73,0.40,0.53}{{#1}}}
    \newcommand{\ImportTok}[1]{{#1}}
    \newcommand{\DocumentationTok}[1]{\textcolor[rgb]{0.73,0.13,0.13}{\textit{{#1}}}}
    \newcommand{\AnnotationTok}[1]{\textcolor[rgb]{0.38,0.63,0.69}{\textbf{\textit{{#1}}}}}
    \newcommand{\CommentVarTok}[1]{\textcolor[rgb]{0.38,0.63,0.69}{\textbf{\textit{{#1}}}}}
    \newcommand{\VariableTok}[1]{\textcolor[rgb]{0.10,0.09,0.49}{{#1}}}
    \newcommand{\ControlFlowTok}[1]{\textcolor[rgb]{0.00,0.44,0.13}{\textbf{{#1}}}}
    \newcommand{\OperatorTok}[1]{\textcolor[rgb]{0.40,0.40,0.40}{{#1}}}
    \newcommand{\BuiltInTok}[1]{{#1}}
    \newcommand{\ExtensionTok}[1]{{#1}}
    \newcommand{\PreprocessorTok}[1]{\textcolor[rgb]{0.74,0.48,0.00}{{#1}}}
    \newcommand{\AttributeTok}[1]{\textcolor[rgb]{0.49,0.56,0.16}{{#1}}}
    \newcommand{\InformationTok}[1]{\textcolor[rgb]{0.38,0.63,0.69}{\textbf{\textit{{#1}}}}}
    \newcommand{\WarningTok}[1]{\textcolor[rgb]{0.38,0.63,0.69}{\textbf{\textit{{#1}}}}}


    % Define a nice break command that doesn't care if a line doesn't already
    % exist.
    \def\br{\hspace*{\fill} \\* }
    % Math Jax compatibility definitions
    \def\gt{>}
    \def\lt{<}
    \let\Oldtex\TeX
    \let\Oldlatex\LaTeX
    \renewcommand{\TeX}{\textrm{\Oldtex}}
    \renewcommand{\LaTeX}{\textrm{\Oldlatex}}
    % Document parameters
    % Document title
    \title{04\_kNNs-SVM-RBF}
    
    
    
    
    
    
    
% Pygments definitions
\makeatletter
\def\PY@reset{\let\PY@it=\relax \let\PY@bf=\relax%
    \let\PY@ul=\relax \let\PY@tc=\relax%
    \let\PY@bc=\relax \let\PY@ff=\relax}
\def\PY@tok#1{\csname PY@tok@#1\endcsname}
\def\PY@toks#1+{\ifx\relax#1\empty\else%
    \PY@tok{#1}\expandafter\PY@toks\fi}
\def\PY@do#1{\PY@bc{\PY@tc{\PY@ul{%
    \PY@it{\PY@bf{\PY@ff{#1}}}}}}}
\def\PY#1#2{\PY@reset\PY@toks#1+\relax+\PY@do{#2}}

\@namedef{PY@tok@w}{\def\PY@tc##1{\textcolor[rgb]{0.73,0.73,0.73}{##1}}}
\@namedef{PY@tok@c}{\let\PY@it=\textit\def\PY@tc##1{\textcolor[rgb]{0.24,0.48,0.48}{##1}}}
\@namedef{PY@tok@cp}{\def\PY@tc##1{\textcolor[rgb]{0.61,0.40,0.00}{##1}}}
\@namedef{PY@tok@k}{\let\PY@bf=\textbf\def\PY@tc##1{\textcolor[rgb]{0.00,0.50,0.00}{##1}}}
\@namedef{PY@tok@kp}{\def\PY@tc##1{\textcolor[rgb]{0.00,0.50,0.00}{##1}}}
\@namedef{PY@tok@kt}{\def\PY@tc##1{\textcolor[rgb]{0.69,0.00,0.25}{##1}}}
\@namedef{PY@tok@o}{\def\PY@tc##1{\textcolor[rgb]{0.40,0.40,0.40}{##1}}}
\@namedef{PY@tok@ow}{\let\PY@bf=\textbf\def\PY@tc##1{\textcolor[rgb]{0.67,0.13,1.00}{##1}}}
\@namedef{PY@tok@nb}{\def\PY@tc##1{\textcolor[rgb]{0.00,0.50,0.00}{##1}}}
\@namedef{PY@tok@nf}{\def\PY@tc##1{\textcolor[rgb]{0.00,0.00,1.00}{##1}}}
\@namedef{PY@tok@nc}{\let\PY@bf=\textbf\def\PY@tc##1{\textcolor[rgb]{0.00,0.00,1.00}{##1}}}
\@namedef{PY@tok@nn}{\let\PY@bf=\textbf\def\PY@tc##1{\textcolor[rgb]{0.00,0.00,1.00}{##1}}}
\@namedef{PY@tok@ne}{\let\PY@bf=\textbf\def\PY@tc##1{\textcolor[rgb]{0.80,0.25,0.22}{##1}}}
\@namedef{PY@tok@nv}{\def\PY@tc##1{\textcolor[rgb]{0.10,0.09,0.49}{##1}}}
\@namedef{PY@tok@no}{\def\PY@tc##1{\textcolor[rgb]{0.53,0.00,0.00}{##1}}}
\@namedef{PY@tok@nl}{\def\PY@tc##1{\textcolor[rgb]{0.46,0.46,0.00}{##1}}}
\@namedef{PY@tok@ni}{\let\PY@bf=\textbf\def\PY@tc##1{\textcolor[rgb]{0.44,0.44,0.44}{##1}}}
\@namedef{PY@tok@na}{\def\PY@tc##1{\textcolor[rgb]{0.41,0.47,0.13}{##1}}}
\@namedef{PY@tok@nt}{\let\PY@bf=\textbf\def\PY@tc##1{\textcolor[rgb]{0.00,0.50,0.00}{##1}}}
\@namedef{PY@tok@nd}{\def\PY@tc##1{\textcolor[rgb]{0.67,0.13,1.00}{##1}}}
\@namedef{PY@tok@s}{\def\PY@tc##1{\textcolor[rgb]{0.73,0.13,0.13}{##1}}}
\@namedef{PY@tok@sd}{\let\PY@it=\textit\def\PY@tc##1{\textcolor[rgb]{0.73,0.13,0.13}{##1}}}
\@namedef{PY@tok@si}{\let\PY@bf=\textbf\def\PY@tc##1{\textcolor[rgb]{0.64,0.35,0.47}{##1}}}
\@namedef{PY@tok@se}{\let\PY@bf=\textbf\def\PY@tc##1{\textcolor[rgb]{0.67,0.36,0.12}{##1}}}
\@namedef{PY@tok@sr}{\def\PY@tc##1{\textcolor[rgb]{0.64,0.35,0.47}{##1}}}
\@namedef{PY@tok@ss}{\def\PY@tc##1{\textcolor[rgb]{0.10,0.09,0.49}{##1}}}
\@namedef{PY@tok@sx}{\def\PY@tc##1{\textcolor[rgb]{0.00,0.50,0.00}{##1}}}
\@namedef{PY@tok@m}{\def\PY@tc##1{\textcolor[rgb]{0.40,0.40,0.40}{##1}}}
\@namedef{PY@tok@gh}{\let\PY@bf=\textbf\def\PY@tc##1{\textcolor[rgb]{0.00,0.00,0.50}{##1}}}
\@namedef{PY@tok@gu}{\let\PY@bf=\textbf\def\PY@tc##1{\textcolor[rgb]{0.50,0.00,0.50}{##1}}}
\@namedef{PY@tok@gd}{\def\PY@tc##1{\textcolor[rgb]{0.63,0.00,0.00}{##1}}}
\@namedef{PY@tok@gi}{\def\PY@tc##1{\textcolor[rgb]{0.00,0.52,0.00}{##1}}}
\@namedef{PY@tok@gr}{\def\PY@tc##1{\textcolor[rgb]{0.89,0.00,0.00}{##1}}}
\@namedef{PY@tok@ge}{\let\PY@it=\textit}
\@namedef{PY@tok@gs}{\let\PY@bf=\textbf}
\@namedef{PY@tok@ges}{\let\PY@bf=\textbf\let\PY@it=\textit}
\@namedef{PY@tok@gp}{\let\PY@bf=\textbf\def\PY@tc##1{\textcolor[rgb]{0.00,0.00,0.50}{##1}}}
\@namedef{PY@tok@go}{\def\PY@tc##1{\textcolor[rgb]{0.44,0.44,0.44}{##1}}}
\@namedef{PY@tok@gt}{\def\PY@tc##1{\textcolor[rgb]{0.00,0.27,0.87}{##1}}}
\@namedef{PY@tok@err}{\def\PY@bc##1{{\setlength{\fboxsep}{\string -\fboxrule}\fcolorbox[rgb]{1.00,0.00,0.00}{1,1,1}{\strut ##1}}}}
\@namedef{PY@tok@kc}{\let\PY@bf=\textbf\def\PY@tc##1{\textcolor[rgb]{0.00,0.50,0.00}{##1}}}
\@namedef{PY@tok@kd}{\let\PY@bf=\textbf\def\PY@tc##1{\textcolor[rgb]{0.00,0.50,0.00}{##1}}}
\@namedef{PY@tok@kn}{\let\PY@bf=\textbf\def\PY@tc##1{\textcolor[rgb]{0.00,0.50,0.00}{##1}}}
\@namedef{PY@tok@kr}{\let\PY@bf=\textbf\def\PY@tc##1{\textcolor[rgb]{0.00,0.50,0.00}{##1}}}
\@namedef{PY@tok@bp}{\def\PY@tc##1{\textcolor[rgb]{0.00,0.50,0.00}{##1}}}
\@namedef{PY@tok@fm}{\def\PY@tc##1{\textcolor[rgb]{0.00,0.00,1.00}{##1}}}
\@namedef{PY@tok@vc}{\def\PY@tc##1{\textcolor[rgb]{0.10,0.09,0.49}{##1}}}
\@namedef{PY@tok@vg}{\def\PY@tc##1{\textcolor[rgb]{0.10,0.09,0.49}{##1}}}
\@namedef{PY@tok@vi}{\def\PY@tc##1{\textcolor[rgb]{0.10,0.09,0.49}{##1}}}
\@namedef{PY@tok@vm}{\def\PY@tc##1{\textcolor[rgb]{0.10,0.09,0.49}{##1}}}
\@namedef{PY@tok@sa}{\def\PY@tc##1{\textcolor[rgb]{0.73,0.13,0.13}{##1}}}
\@namedef{PY@tok@sb}{\def\PY@tc##1{\textcolor[rgb]{0.73,0.13,0.13}{##1}}}
\@namedef{PY@tok@sc}{\def\PY@tc##1{\textcolor[rgb]{0.73,0.13,0.13}{##1}}}
\@namedef{PY@tok@dl}{\def\PY@tc##1{\textcolor[rgb]{0.73,0.13,0.13}{##1}}}
\@namedef{PY@tok@s2}{\def\PY@tc##1{\textcolor[rgb]{0.73,0.13,0.13}{##1}}}
\@namedef{PY@tok@sh}{\def\PY@tc##1{\textcolor[rgb]{0.73,0.13,0.13}{##1}}}
\@namedef{PY@tok@s1}{\def\PY@tc##1{\textcolor[rgb]{0.73,0.13,0.13}{##1}}}
\@namedef{PY@tok@mb}{\def\PY@tc##1{\textcolor[rgb]{0.40,0.40,0.40}{##1}}}
\@namedef{PY@tok@mf}{\def\PY@tc##1{\textcolor[rgb]{0.40,0.40,0.40}{##1}}}
\@namedef{PY@tok@mh}{\def\PY@tc##1{\textcolor[rgb]{0.40,0.40,0.40}{##1}}}
\@namedef{PY@tok@mi}{\def\PY@tc##1{\textcolor[rgb]{0.40,0.40,0.40}{##1}}}
\@namedef{PY@tok@il}{\def\PY@tc##1{\textcolor[rgb]{0.40,0.40,0.40}{##1}}}
\@namedef{PY@tok@mo}{\def\PY@tc##1{\textcolor[rgb]{0.40,0.40,0.40}{##1}}}
\@namedef{PY@tok@ch}{\let\PY@it=\textit\def\PY@tc##1{\textcolor[rgb]{0.24,0.48,0.48}{##1}}}
\@namedef{PY@tok@cm}{\let\PY@it=\textit\def\PY@tc##1{\textcolor[rgb]{0.24,0.48,0.48}{##1}}}
\@namedef{PY@tok@cpf}{\let\PY@it=\textit\def\PY@tc##1{\textcolor[rgb]{0.24,0.48,0.48}{##1}}}
\@namedef{PY@tok@c1}{\let\PY@it=\textit\def\PY@tc##1{\textcolor[rgb]{0.24,0.48,0.48}{##1}}}
\@namedef{PY@tok@cs}{\let\PY@it=\textit\def\PY@tc##1{\textcolor[rgb]{0.24,0.48,0.48}{##1}}}

\def\PYZbs{\char`\\}
\def\PYZus{\char`\_}
\def\PYZob{\char`\{}
\def\PYZcb{\char`\}}
\def\PYZca{\char`\^}
\def\PYZam{\char`\&}
\def\PYZlt{\char`\<}
\def\PYZgt{\char`\>}
\def\PYZsh{\char`\#}
\def\PYZpc{\char`\%}
\def\PYZdl{\char`\$}
\def\PYZhy{\char`\-}
\def\PYZsq{\char`\'}
\def\PYZdq{\char`\"}
\def\PYZti{\char`\~}
% for compatibility with earlier versions
\def\PYZat{@}
\def\PYZlb{[}
\def\PYZrb{]}
\makeatother


    % For linebreaks inside Verbatim environment from package fancyvrb.
    \makeatletter
        \newbox\Wrappedcontinuationbox
        \newbox\Wrappedvisiblespacebox
        \newcommand*\Wrappedvisiblespace {\textcolor{red}{\textvisiblespace}}
        \newcommand*\Wrappedcontinuationsymbol {\textcolor{red}{\llap{\tiny$\m@th\hookrightarrow$}}}
        \newcommand*\Wrappedcontinuationindent {3ex }
        \newcommand*\Wrappedafterbreak {\kern\Wrappedcontinuationindent\copy\Wrappedcontinuationbox}
        % Take advantage of the already applied Pygments mark-up to insert
        % potential linebreaks for TeX processing.
        %        {, <, #, %, $, ' and ": go to next line.
        %        _, }, ^, &, >, - and ~: stay at end of broken line.
        % Use of \textquotesingle for straight quote.
        \newcommand*\Wrappedbreaksatspecials {%
            \def\PYGZus{\discretionary{\char`\_}{\Wrappedafterbreak}{\char`\_}}%
            \def\PYGZob{\discretionary{}{\Wrappedafterbreak\char`\{}{\char`\{}}%
            \def\PYGZcb{\discretionary{\char`\}}{\Wrappedafterbreak}{\char`\}}}%
            \def\PYGZca{\discretionary{\char`\^}{\Wrappedafterbreak}{\char`\^}}%
            \def\PYGZam{\discretionary{\char`\&}{\Wrappedafterbreak}{\char`\&}}%
            \def\PYGZlt{\discretionary{}{\Wrappedafterbreak\char`\<}{\char`\<}}%
            \def\PYGZgt{\discretionary{\char`\>}{\Wrappedafterbreak}{\char`\>}}%
            \def\PYGZsh{\discretionary{}{\Wrappedafterbreak\char`\#}{\char`\#}}%
            \def\PYGZpc{\discretionary{}{\Wrappedafterbreak\char`\%}{\char`\%}}%
            \def\PYGZdl{\discretionary{}{\Wrappedafterbreak\char`\$}{\char`\$}}%
            \def\PYGZhy{\discretionary{\char`\-}{\Wrappedafterbreak}{\char`\-}}%
            \def\PYGZsq{\discretionary{}{\Wrappedafterbreak\textquotesingle}{\textquotesingle}}%
            \def\PYGZdq{\discretionary{}{\Wrappedafterbreak\char`\"}{\char`\"}}%
            \def\PYGZti{\discretionary{\char`\~}{\Wrappedafterbreak}{\char`\~}}%
        }
        % Some characters . , ; ? ! / are not pygmentized.
        % This macro makes them "active" and they will insert potential linebreaks
        \newcommand*\Wrappedbreaksatpunct {%
            \lccode`\~`\.\lowercase{\def~}{\discretionary{\hbox{\char`\.}}{\Wrappedafterbreak}{\hbox{\char`\.}}}%
            \lccode`\~`\,\lowercase{\def~}{\discretionary{\hbox{\char`\,}}{\Wrappedafterbreak}{\hbox{\char`\,}}}%
            \lccode`\~`\;\lowercase{\def~}{\discretionary{\hbox{\char`\;}}{\Wrappedafterbreak}{\hbox{\char`\;}}}%
            \lccode`\~`\:\lowercase{\def~}{\discretionary{\hbox{\char`\:}}{\Wrappedafterbreak}{\hbox{\char`\:}}}%
            \lccode`\~`\?\lowercase{\def~}{\discretionary{\hbox{\char`\?}}{\Wrappedafterbreak}{\hbox{\char`\?}}}%
            \lccode`\~`\!\lowercase{\def~}{\discretionary{\hbox{\char`\!}}{\Wrappedafterbreak}{\hbox{\char`\!}}}%
            \lccode`\~`\/\lowercase{\def~}{\discretionary{\hbox{\char`\/}}{\Wrappedafterbreak}{\hbox{\char`\/}}}%
            \catcode`\.\active
            \catcode`\,\active
            \catcode`\;\active
            \catcode`\:\active
            \catcode`\?\active
            \catcode`\!\active
            \catcode`\/\active
            \lccode`\~`\~
        }
    \makeatother

    \let\OriginalVerbatim=\Verbatim
    \makeatletter
    \renewcommand{\Verbatim}[1][1]{%
        %\parskip\z@skip
        \sbox\Wrappedcontinuationbox {\Wrappedcontinuationsymbol}%
        \sbox\Wrappedvisiblespacebox {\FV@SetupFont\Wrappedvisiblespace}%
        \def\FancyVerbFormatLine ##1{\hsize\linewidth
            \vtop{\raggedright\hyphenpenalty\z@\exhyphenpenalty\z@
                \doublehyphendemerits\z@\finalhyphendemerits\z@
                \strut ##1\strut}%
        }%
        % If the linebreak is at a space, the latter will be displayed as visible
        % space at end of first line, and a continuation symbol starts next line.
        % Stretch/shrink are however usually zero for typewriter font.
        \def\FV@Space {%
            \nobreak\hskip\z@ plus\fontdimen3\font minus\fontdimen4\font
            \discretionary{\copy\Wrappedvisiblespacebox}{\Wrappedafterbreak}
            {\kern\fontdimen2\font}%
        }%

        % Allow breaks at special characters using \PYG... macros.
        \Wrappedbreaksatspecials
        % Breaks at punctuation characters . , ; ? ! and / need catcode=\active
        \OriginalVerbatim[#1,codes*=\Wrappedbreaksatpunct]%
    }
    \makeatother

    % Exact colors from NB
    \definecolor{incolor}{HTML}{303F9F}
    \definecolor{outcolor}{HTML}{D84315}
    \definecolor{cellborder}{HTML}{CFCFCF}
    \definecolor{cellbackground}{HTML}{F7F7F7}

    % prompt
    \makeatletter
    \newcommand{\boxspacing}{\kern\kvtcb@left@rule\kern\kvtcb@boxsep}
    \makeatother
    \newcommand{\prompt}[4]{
        {\ttfamily\llap{{\color{#2}[#3]:\hspace{3pt}#4}}\vspace{-\baselineskip}}
    }
    

    
    % Prevent overflowing lines due to hard-to-break entities
    \sloppy
    % Setup hyperref package
    \hypersetup{
      breaklinks=true,  % so long urls are correctly broken across lines
      colorlinks=true,
      urlcolor=urlcolor,
      linkcolor=linkcolor,
      citecolor=citecolor,
      }
    % Slightly bigger margins than the latex defaults
    
    \geometry{verbose,tmargin=1in,bmargin=1in,lmargin=1in,rmargin=1in}
    
    

\begin{document}
    
    \maketitle
    
    

    
    \includegraphics{/Users/charleenchu/Desktop/school/cpsc330-2023W2/lectures/img/330-banner.png}

    \section{\texorpdfstring{Lecture 4: \(k\)-Nearest Neighbours and SVM
RBFs}{Lecture 4: k-Nearest Neighbours and SVM RBFs}}\label{lecture-4-k-nearest-neighbours-and-svm-rbfs}

UBC 2023-24

Instructors: Mathias Lécuyer and Mehrdad Oveisi

\begin{quote}
If two things are similar, the thought of one will tend to trigger the
thought of the other -- Aristotle
\end{quote}

    \subsubsection{Announcements}\label{announcements}

\begin{itemize}
\tightlist
\item
  hw2 was due Yesterday.
\item
  hw3 will be released soon, you have more time to do it. ((Due Monday,
  Feb 5 at
  11:59pm.){[}https://github.com/UBC-CS/cpsc330-2023W2\#deliverable-due-dates-tentative{]})

  \begin{itemize}
  \tightlist
  \item
    We are allowing group submissions for this homework.
  \end{itemize}
\item
  The lecture notes within these notebooks align with the content
  presented in the videos. Even though we do not cover all the content
  from these notebooks during lectures, it's your responsibility to go
  through them on your own.
\item
  By this point, you should know your course enrollment status.
  Registration for tutorials is not mandatory; they are optional and
  will follow an office-hour format. You are free to attend any tutorial
  session of your choice.
\end{itemize}

    \subsection{Imports, and LOs}\label{imports-and-los}

    \subsubsection{Imports}\label{imports}

    \begin{tcolorbox}[breakable, size=fbox, boxrule=1pt, pad at break*=1mm,colback=cellbackground, colframe=cellborder]
\prompt{In}{incolor}{1}{\boxspacing}
\begin{Verbatim}[commandchars=\\\{\}]
\PY{k+kn}{import} \PY{n+nn}{sys}

\PY{k+kn}{import} \PY{n+nn}{IPython}
\PY{k+kn}{import} \PY{n+nn}{matplotlib}\PY{n+nn}{.}\PY{n+nn}{pyplot} \PY{k}{as} \PY{n+nn}{plt}
\PY{k+kn}{import} \PY{n+nn}{numpy} \PY{k}{as} \PY{n+nn}{np}
\PY{k+kn}{import} \PY{n+nn}{pandas} \PY{k}{as} \PY{n+nn}{pd}
\PY{k+kn}{from} \PY{n+nn}{IPython}\PY{n+nn}{.}\PY{n+nn}{display} \PY{k+kn}{import} \PY{n}{HTML}

\PY{n}{sys}\PY{o}{.}\PY{n}{path}\PY{o}{.}\PY{n}{append}\PY{p}{(}\PY{l+s+s2}{\PYZdq{}}\PY{l+s+s2}{../code/.}\PY{l+s+s2}{\PYZdq{}}\PY{p}{)}

\PY{k+kn}{import} \PY{n+nn}{ipywidgets} \PY{k}{as} \PY{n+nn}{widgets}
\PY{k+kn}{import} \PY{n+nn}{mglearn\PYZus{}utils}
\PY{k+kn}{from} \PY{n+nn}{IPython}\PY{n+nn}{.}\PY{n+nn}{display} \PY{k+kn}{import} \PY{n}{display}
\PY{k+kn}{from} \PY{n+nn}{ipywidgets} \PY{k+kn}{import} \PY{n}{interact}\PY{p}{,} \PY{n}{interactive}
\PY{k+kn}{from} \PY{n+nn}{plotting\PYZus{}functions} \PY{k+kn}{import} \PY{o}{*}
\PY{k+kn}{from} \PY{n+nn}{sklearn}\PY{n+nn}{.}\PY{n+nn}{dummy} \PY{k+kn}{import} \PY{n}{DummyClassifier}
\PY{k+kn}{from} \PY{n+nn}{sklearn}\PY{n+nn}{.}\PY{n+nn}{model\PYZus{}selection} \PY{k+kn}{import} \PY{n}{cross\PYZus{}validate}\PY{p}{,} \PY{n}{train\PYZus{}test\PYZus{}split}
\PY{k+kn}{from} \PY{n+nn}{utils} \PY{k+kn}{import} \PY{o}{*}

\PY{o}{\PYZpc{}}\PY{k}{matplotlib} inline

\PY{n}{pd}\PY{o}{.}\PY{n}{set\PYZus{}option}\PY{p}{(}\PY{l+s+s2}{\PYZdq{}}\PY{l+s+s2}{display.max\PYZus{}colwidth}\PY{l+s+s2}{\PYZdq{}}\PY{p}{,} \PY{l+m+mi}{200}\PY{p}{)}
\PY{k+kn}{import} \PY{n+nn}{warnings}

\PY{n}{warnings}\PY{o}{.}\PY{n}{filterwarnings}\PY{p}{(}\PY{l+s+s2}{\PYZdq{}}\PY{l+s+s2}{ignore}\PY{l+s+s2}{\PYZdq{}}\PY{p}{)}
\end{Verbatim}
\end{tcolorbox}

    

    \subsubsection{Learning outcomes}\label{learning-outcomes}

From this lecture, you will be able to

\begin{itemize}
\tightlist
\item
  explain the notion of similarity-based algorithms;
\item
  broadly describe how \(k\)-NNs use distances;
\item
  discuss the effect of using a small/large value of the hyperparameter
  \(k\) when using the \(k\)-NN algorithm;
\item
  describe the problem of curse of dimensionality;
\item
  explain the general idea of SVMs with RBF kernel;
\item
  broadly describe the relation of \texttt{gamma} and \texttt{C}
  hyperparameters of SVMs with the fundamental tradeoff.
\end{itemize}

    

    \subsubsection{Quick recap}\label{quick-recap}

\begin{itemize}
\tightlist
\item
  Why do we split the data?
\item
  What are the 4 types of data we discussed last class?
\item
  What are the benefits of cross-validation?
\item
  What is overfitting?
\item
  What's the fundamental trade-off in supervised machine learning?
\item
  What is the golden rule of machine learning?
\end{itemize}

    

    \subsection{\texorpdfstring{Motivation and distances
{[}\href{https://youtu.be/hCa3EXEUmQk}{video}{]}}{Motivation and distances {[}video{]}}}\label{motivation-and-distances-video}

    \subsubsection{Analogy-based models}\label{analogy-based-models}

\begin{itemize}
\tightlist
\item
  Suppose you are given the following training examples with
  corresponding labels and are asked to label a given test example.
\end{itemize}

\href{https://vipl.ict.ac.cn/en/database.php}{source}

\begin{itemize}
\tightlist
\item
  An intuitive way to classify the test example is by finding the most
  ``similar'' example(s) from the training set and using that label for
  the test example.
\end{itemize}

    \subsubsection{Analogy-based algorithms in
practice}\label{analogy-based-algorithms-in-practice}

\begin{itemize}
\tightlist
\item
  \href{https://www.hertasecurity.com/en}{Herta's High-tech Facial
  Recognition}

  \begin{itemize}
  \tightlist
  \item
    Feature vectors for human faces
  \item
    \(k\)-NN to identify which face is on their watch list
  \end{itemize}
\item
  Recommendation systems
\end{itemize}

    \subsubsection{\texorpdfstring{General idea of \(k\)-nearest neighbours
algorithm}{General idea of k-nearest neighbours algorithm}}\label{general-idea-of-k-nearest-neighbours-algorithm}

\begin{itemize}
\tightlist
\item
  Consider the following toy dataset with two classes.

  \begin{itemize}
  \tightlist
  \item
    blue circles \(\rightarrow\) class 0
  \item
    red triangles \(\rightarrow\) class 1
  \item
    green stars \(\rightarrow\) test examples
  \end{itemize}
\end{itemize}

    \begin{tcolorbox}[breakable, size=fbox, boxrule=1pt, pad at break*=1mm,colback=cellbackground, colframe=cellborder]
\prompt{In}{incolor}{2}{\boxspacing}
\begin{Verbatim}[commandchars=\\\{\}]
\PY{n}{X}\PY{p}{,} \PY{n}{y} \PY{o}{=} \PY{n}{mglearn\PYZus{}utils}\PY{o}{.}\PY{n}{make\PYZus{}forge}\PY{p}{(}\PY{p}{)}
\PY{n}{X\PYZus{}test} \PY{o}{=} \PY{n}{np}\PY{o}{.}\PY{n}{array}\PY{p}{(}\PY{p}{[}\PY{p}{[}\PY{l+m+mf}{8.2}\PY{p}{,} \PY{l+m+mf}{3.66214339}\PY{p}{]}\PY{p}{,} \PY{p}{[}\PY{l+m+mf}{9.9}\PY{p}{,} \PY{l+m+mf}{3.2}\PY{p}{]}\PY{p}{,} \PY{p}{[}\PY{l+m+mf}{11.2}\PY{p}{,} \PY{l+m+mf}{0.5}\PY{p}{]}\PY{p}{]}\PY{p}{)}
\end{Verbatim}
\end{tcolorbox}

    \begin{tcolorbox}[breakable, size=fbox, boxrule=1pt, pad at break*=1mm,colback=cellbackground, colframe=cellborder]
\prompt{In}{incolor}{3}{\boxspacing}
\begin{Verbatim}[commandchars=\\\{\}]
\PY{n}{plot\PYZus{}train\PYZus{}test\PYZus{}points}\PY{p}{(}\PY{n}{X}\PY{p}{,} \PY{n}{y}\PY{p}{,} \PY{n}{X\PYZus{}test}\PY{p}{)}
\end{Verbatim}
\end{tcolorbox}

    \begin{center}
    \adjustimage{max size={0.9\linewidth}{0.9\paperheight}}{output_16_0.png}
    \end{center}
    { \hspace*{\fill} \\}
    
    \begin{itemize}
\tightlist
\item
  Given a new data point, predict the class of the data point by finding
  the ``closest'' data point in the training set, i.e., by finding its
  ``nearest neighbour'' or majority vote of nearest neighbours.
\end{itemize}

    \begin{tcolorbox}[breakable, size=fbox, boxrule=1pt, pad at break*=1mm,colback=cellbackground, colframe=cellborder]
\prompt{In}{incolor}{4}{\boxspacing}
\begin{Verbatim}[commandchars=\\\{\}]
\PY{k}{def} \PY{n+nf}{f}\PY{p}{(}\PY{n}{n\PYZus{}neighbors}\PY{p}{)}\PY{p}{:}
    \PY{k}{return} \PY{n}{plot\PYZus{}knn\PYZus{}clf}\PY{p}{(}\PY{n}{X}\PY{p}{,} \PY{n}{y}\PY{p}{,} \PY{n}{X\PYZus{}test}\PY{p}{,} \PY{n}{n\PYZus{}neighbors}\PY{o}{=}\PY{n}{n\PYZus{}neighbors}\PY{p}{)}
\end{Verbatim}
\end{tcolorbox}

    \begin{tcolorbox}[breakable, size=fbox, boxrule=1pt, pad at break*=1mm,colback=cellbackground, colframe=cellborder]
\prompt{In}{incolor}{8}{\boxspacing}
\begin{Verbatim}[commandchars=\\\{\}]
\PY{n}{f}\PY{p}{(}\PY{l+m+mi}{3}\PY{p}{)}
\end{Verbatim}
\end{tcolorbox}

    \begin{Verbatim}[commandchars=\\\{\}]
n\_neighbors 3
    \end{Verbatim}

    \begin{center}
    \adjustimage{max size={0.9\linewidth}{0.9\paperheight}}{output_19_1.png}
    \end{center}
    { \hspace*{\fill} \\}
    
    \subsubsection{Geometric view of tabular data and
dimensions}\label{geometric-view-of-tabular-data-and-dimensions}

\begin{itemize}
\tightlist
\item
  To understand analogy-based algorithms it's useful to think of data as
  points in a high dimensional space.
\item
  Our \texttt{X} represents the problem in terms of relevant
  \textbf{features} (\(d\)) with one dimension for each \textbf{feature}
  (column).
\item
  Examples are \textbf{points in a \(d\)-dimensional space}.
\end{itemize}

    How many dimensions (features) are there in the cities data?

    \begin{tcolorbox}[breakable, size=fbox, boxrule=1pt, pad at break*=1mm,colback=cellbackground, colframe=cellborder]
\prompt{In}{incolor}{6}{\boxspacing}
\begin{Verbatim}[commandchars=\\\{\}]
\PY{n}{cities\PYZus{}df} \PY{o}{=} \PY{n}{pd}\PY{o}{.}\PY{n}{read\PYZus{}csv}\PY{p}{(}\PY{l+s+s2}{\PYZdq{}}\PY{l+s+s2}{../data/canada\PYZus{}usa\PYZus{}cities.csv}\PY{l+s+s2}{\PYZdq{}}\PY{p}{)}
\PY{n}{X\PYZus{}cities} \PY{o}{=} \PY{n}{cities\PYZus{}df}\PY{p}{[}\PY{p}{[}\PY{l+s+s2}{\PYZdq{}}\PY{l+s+s2}{longitude}\PY{l+s+s2}{\PYZdq{}}\PY{p}{,} \PY{l+s+s2}{\PYZdq{}}\PY{l+s+s2}{latitude}\PY{l+s+s2}{\PYZdq{}}\PY{p}{]}\PY{p}{]}
\PY{n}{y\PYZus{}cities} \PY{o}{=} \PY{n}{cities\PYZus{}df}\PY{p}{[}\PY{l+s+s2}{\PYZdq{}}\PY{l+s+s2}{country}\PY{l+s+s2}{\PYZdq{}}\PY{p}{]}
\end{Verbatim}
\end{tcolorbox}

    \begin{tcolorbox}[breakable, size=fbox, boxrule=1pt, pad at break*=1mm,colback=cellbackground, colframe=cellborder]
\prompt{In}{incolor}{7}{\boxspacing}
\begin{Verbatim}[commandchars=\\\{\}]
\PY{n}{mglearn\PYZus{}utils}\PY{o}{.}\PY{n}{discrete\PYZus{}scatter}\PY{p}{(}\PY{n}{X\PYZus{}cities}\PY{o}{.}\PY{n}{iloc}\PY{p}{[}\PY{p}{:}\PY{p}{,} \PY{l+m+mi}{0}\PY{p}{]}\PY{p}{,} \PY{n}{X\PYZus{}cities}\PY{o}{.}\PY{n}{iloc}\PY{p}{[}\PY{p}{:}\PY{p}{,} \PY{l+m+mi}{1}\PY{p}{]}\PY{p}{,} \PY{n}{y\PYZus{}cities}\PY{p}{)}
\PY{n}{plt}\PY{o}{.}\PY{n}{xlabel}\PY{p}{(}\PY{l+s+s2}{\PYZdq{}}\PY{l+s+s2}{longitude}\PY{l+s+s2}{\PYZdq{}}\PY{p}{)}
\PY{n}{plt}\PY{o}{.}\PY{n}{ylabel}\PY{p}{(}\PY{l+s+s2}{\PYZdq{}}\PY{l+s+s2}{latitude}\PY{l+s+s2}{\PYZdq{}}\PY{p}{)}\PY{p}{;}
\end{Verbatim}
\end{tcolorbox}

    \begin{center}
    \adjustimage{max size={0.9\linewidth}{0.9\paperheight}}{output_23_0.png}
    \end{center}
    { \hspace*{\fill} \\}
    
    \begin{itemize}
\tightlist
\item
  Recall the
  \href{https://www.kaggle.com/geomack/spotifyclassification/home}{Spotify
  Song Attributes} dataset from homework 1.
\item
  How many dimensions (features) we used in the homework?
\end{itemize}

    \begin{tcolorbox}[breakable, size=fbox, boxrule=1pt, pad at break*=1mm,colback=cellbackground, colframe=cellborder]
\prompt{In}{incolor}{8}{\boxspacing}
\begin{Verbatim}[commandchars=\\\{\}]
\PY{n}{spotify\PYZus{}df} \PY{o}{=} \PY{n}{pd}\PY{o}{.}\PY{n}{read\PYZus{}csv}\PY{p}{(}\PY{l+s+s2}{\PYZdq{}}\PY{l+s+s2}{../data/spotify.csv}\PY{l+s+s2}{\PYZdq{}}\PY{p}{,} \PY{n}{index\PYZus{}col}\PY{o}{=}\PY{l+m+mi}{0}\PY{p}{)}
\PY{n}{X\PYZus{}spotify} \PY{o}{=} \PY{n}{spotify\PYZus{}df}\PY{o}{.}\PY{n}{drop}\PY{p}{(}\PY{n}{columns}\PY{o}{=}\PY{p}{[}\PY{l+s+s2}{\PYZdq{}}\PY{l+s+s2}{target}\PY{l+s+s2}{\PYZdq{}}\PY{p}{,} \PY{l+s+s2}{\PYZdq{}}\PY{l+s+s2}{song\PYZus{}title}\PY{l+s+s2}{\PYZdq{}}\PY{p}{,} \PY{l+s+s2}{\PYZdq{}}\PY{l+s+s2}{artist}\PY{l+s+s2}{\PYZdq{}}\PY{p}{]}\PY{p}{)}
\PY{n+nb}{print}\PY{p}{(}\PY{l+s+s2}{\PYZdq{}}\PY{l+s+s2}{The number of features in the Spotify dataset: }\PY{l+s+si}{\PYZpc{}d}\PY{l+s+s2}{\PYZdq{}} \PY{o}{\PYZpc{}} \PY{n}{X\PYZus{}spotify}\PY{o}{.}\PY{n}{shape}\PY{p}{[}\PY{l+m+mi}{1}\PY{p}{]}\PY{p}{)}
\PY{n}{X\PYZus{}spotify}\PY{o}{.}\PY{n}{head}\PY{p}{(}\PY{p}{)}
\end{Verbatim}
\end{tcolorbox}

    \begin{Verbatim}[commandchars=\\\{\}]
The number of features in the Spotify dataset: 13
    \end{Verbatim}

            \begin{tcolorbox}[breakable, size=fbox, boxrule=.5pt, pad at break*=1mm, opacityfill=0]
\prompt{Out}{outcolor}{8}{\boxspacing}
\begin{Verbatim}[commandchars=\\\{\}]
   acousticness  danceability  duration\_ms  energy  instrumentalness  key  \textbackslash{}
0        0.0102         0.833       204600   0.434          0.021900    2
1        0.1990         0.743       326933   0.359          0.006110    1
2        0.0344         0.838       185707   0.412          0.000234    2
3        0.6040         0.494       199413   0.338          0.510000    5
4        0.1800         0.678       392893   0.561          0.512000    5

   liveness  loudness  mode  speechiness    tempo  time\_signature  valence
0    0.1650    -8.795     1       0.4310  150.062             4.0    0.286
1    0.1370   -10.401     1       0.0794  160.083             4.0    0.588
2    0.1590    -7.148     1       0.2890   75.044             4.0    0.173
3    0.0922   -15.236     1       0.0261   86.468             4.0    0.230
4    0.4390   -11.648     0       0.0694  174.004             4.0    0.904
\end{Verbatim}
\end{tcolorbox}
        
    \subsubsection{Dimensions in ML
problems}\label{dimensions-in-ml-problems}

In ML, usually we deal with high dimensional problems where examples are
hard to visualize.

\begin{itemize}
\tightlist
\item
  \(d \approx 20\) is considered low dimensional
\item
  \(d \approx 1000\) is considered medium dimensional
\item
  \(d \approx 100,000\) is considered high dimensional
\end{itemize}

    \subsubsection{Feature vectors}\label{feature-vectors}

\begin{description}
\tightlist
\item[\textbf{Feature vector}]
is composed of feature values associated with an example.
\end{description}

Some example feature vectors are shown below.

    \begin{tcolorbox}[breakable, size=fbox, boxrule=1pt, pad at break*=1mm,colback=cellbackground, colframe=cellborder]
\prompt{In}{incolor}{9}{\boxspacing}
\begin{Verbatim}[commandchars=\\\{\}]
\PY{n+nb}{print}\PY{p}{(}
    \PY{l+s+s2}{\PYZdq{}}\PY{l+s+s2}{An example feature vector from the cities dataset: }\PY{l+s+si}{\PYZpc{}s}\PY{l+s+s2}{\PYZdq{}}
    \PY{o}{\PYZpc{}} \PY{p}{(}\PY{n}{X\PYZus{}cities}\PY{o}{.}\PY{n}{iloc}\PY{p}{[}\PY{l+m+mi}{0}\PY{p}{]}\PY{o}{.}\PY{n}{to\PYZus{}numpy}\PY{p}{(}\PY{p}{)}\PY{p}{)}
\PY{p}{)}
\PY{n+nb}{print}\PY{p}{(}
    \PY{l+s+s2}{\PYZdq{}}\PY{l+s+s2}{An example feature vector from the Spotify dataset: }\PY{l+s+se}{\PYZbs{}n}\PY{l+s+si}{\PYZpc{}s}\PY{l+s+s2}{\PYZdq{}}
    \PY{o}{\PYZpc{}} \PY{p}{(}\PY{n}{X\PYZus{}spotify}\PY{o}{.}\PY{n}{iloc}\PY{p}{[}\PY{l+m+mi}{0}\PY{p}{]}\PY{o}{.}\PY{n}{to\PYZus{}numpy}\PY{p}{(}\PY{p}{)}\PY{p}{)}
\PY{p}{)}
\end{Verbatim}
\end{tcolorbox}

    \begin{Verbatim}[commandchars=\\\{\}]
An example feature vector from the cities dataset: [-130.0437   55.9773]
An example feature vector from the Spotify dataset:
[ 1.02000e-02  8.33000e-01  2.04600e+05  4.34000e-01  2.19000e-02
  2.00000e+00  1.65000e-01 -8.79500e+00  1.00000e+00  4.31000e-01
  1.50062e+02  4.00000e+00  2.86000e-01]
    \end{Verbatim}

    \subsubsection{Similarity between
examples}\label{similarity-between-examples}

Let's take 2 points (two feature vectors) from the cities dataset.

    \begin{tcolorbox}[breakable, size=fbox, boxrule=1pt, pad at break*=1mm,colback=cellbackground, colframe=cellborder]
\prompt{In}{incolor}{10}{\boxspacing}
\begin{Verbatim}[commandchars=\\\{\}]
\PY{n}{two\PYZus{}cities} \PY{o}{=} \PY{n}{X\PYZus{}cities}\PY{o}{.}\PY{n}{sample}\PY{p}{(}\PY{l+m+mi}{2}\PY{p}{,} \PY{n}{random\PYZus{}state}\PY{o}{=}\PY{l+m+mi}{120}\PY{p}{)}
\PY{n}{two\PYZus{}cities}
\end{Verbatim}
\end{tcolorbox}

            \begin{tcolorbox}[breakable, size=fbox, boxrule=.5pt, pad at break*=1mm, opacityfill=0]
\prompt{Out}{outcolor}{10}{\boxspacing}
\begin{Verbatim}[commandchars=\\\{\}]
    longitude  latitude
69  -104.8253   38.8340
35  -112.0741   33.4484
\end{Verbatim}
\end{tcolorbox}
        
    The two sampled points are shown as big black circles.

    \begin{tcolorbox}[breakable, size=fbox, boxrule=1pt, pad at break*=1mm,colback=cellbackground, colframe=cellborder]
\prompt{In}{incolor}{11}{\boxspacing}
\begin{Verbatim}[commandchars=\\\{\}]
\PY{n}{mglearn\PYZus{}utils}\PY{o}{.}\PY{n}{discrete\PYZus{}scatter}\PY{p}{(}
    \PY{n}{X\PYZus{}cities}\PY{o}{.}\PY{n}{iloc}\PY{p}{[}\PY{p}{:}\PY{p}{,} \PY{l+m+mi}{0}\PY{p}{]}\PY{p}{,} \PY{n}{X\PYZus{}cities}\PY{o}{.}\PY{n}{iloc}\PY{p}{[}\PY{p}{:}\PY{p}{,} \PY{l+m+mi}{1}\PY{p}{]}\PY{p}{,} \PY{n}{y\PYZus{}cities}\PY{p}{,} \PY{n}{s}\PY{o}{=}\PY{l+m+mi}{8}\PY{p}{,} \PY{n}{alpha}\PY{o}{=}\PY{l+m+mf}{0.3}
\PY{p}{)}
\PY{n}{mglearn\PYZus{}utils}\PY{o}{.}\PY{n}{discrete\PYZus{}scatter}\PY{p}{(}
    \PY{n}{two\PYZus{}cities}\PY{o}{.}\PY{n}{iloc}\PY{p}{[}\PY{p}{:}\PY{p}{,} \PY{l+m+mi}{0}\PY{p}{]}\PY{p}{,} \PY{n}{two\PYZus{}cities}\PY{o}{.}\PY{n}{iloc}\PY{p}{[}\PY{p}{:}\PY{p}{,} \PY{l+m+mi}{1}\PY{p}{]}\PY{p}{,} \PY{n}{markers}\PY{o}{=}\PY{l+s+s2}{\PYZdq{}}\PY{l+s+s2}{o}\PY{l+s+s2}{\PYZdq{}}\PY{p}{,} \PY{n}{c}\PY{o}{=}\PY{l+s+s2}{\PYZdq{}}\PY{l+s+s2}{k}\PY{l+s+s2}{\PYZdq{}}\PY{p}{,} \PY{n}{s}\PY{o}{=}\PY{l+m+mi}{18}
\PY{p}{)}\PY{p}{;}
\end{Verbatim}
\end{tcolorbox}

    \begin{center}
    \adjustimage{max size={0.9\linewidth}{0.9\paperheight}}{output_32_0.png}
    \end{center}
    { \hspace*{\fill} \\}
    
    \subsubsection{Distance between feature
vectors}\label{distance-between-feature-vectors}

\begin{itemize}
\tightlist
\item
  For the cities at the two big circles, what is the \emph{distance}
  between them?
\item
  A common way to calculate the distance between vectors is calculating
  the \textbf{Euclidean distance}.
\item
  The euclidean distance between vectors \(u = <u_1, u_2, \dots, u_n>\)
  and \(v = <v_1, v_2, \dots, v_n>\) is defined as:
\end{itemize}

\[distance(u, v) = \sqrt{\sum_{i =1}^{n} (u_i - v_i)^2}\]

    \subsubsection{Euclidean distance}\label{euclidean-distance}

    \begin{tcolorbox}[breakable, size=fbox, boxrule=1pt, pad at break*=1mm,colback=cellbackground, colframe=cellborder]
\prompt{In}{incolor}{12}{\boxspacing}
\begin{Verbatim}[commandchars=\\\{\}]
\PY{n}{two\PYZus{}cities}
\end{Verbatim}
\end{tcolorbox}

            \begin{tcolorbox}[breakable, size=fbox, boxrule=.5pt, pad at break*=1mm, opacityfill=0]
\prompt{Out}{outcolor}{12}{\boxspacing}
\begin{Verbatim}[commandchars=\\\{\}]
    longitude  latitude
69  -104.8253   38.8340
35  -112.0741   33.4484
\end{Verbatim}
\end{tcolorbox}
        
    \begin{itemize}
\tightlist
\item
  Subtract the two cities
\item
  Square the difference
\item
  Sum them up
\item
  Take the square root
\end{itemize}

    \begin{tcolorbox}[breakable, size=fbox, boxrule=1pt, pad at break*=1mm,colback=cellbackground, colframe=cellborder]
\prompt{In}{incolor}{13}{\boxspacing}
\begin{Verbatim}[commandchars=\\\{\}]
\PY{c+c1}{\PYZsh{} Subtract the two cities}
\PY{n+nb}{print}\PY{p}{(}\PY{l+s+s2}{\PYZdq{}}\PY{l+s+s2}{Subtract the cities: }\PY{l+s+se}{\PYZbs{}n}\PY{l+s+si}{\PYZpc{}s}\PY{l+s+se}{\PYZbs{}n}\PY{l+s+s2}{\PYZdq{}} \PY{o}{\PYZpc{}} \PY{p}{(}\PY{n}{two\PYZus{}cities}\PY{o}{.}\PY{n}{iloc}\PY{p}{[}\PY{l+m+mi}{1}\PY{p}{]} \PY{o}{\PYZhy{}} \PY{n}{two\PYZus{}cities}\PY{o}{.}\PY{n}{iloc}\PY{p}{[}\PY{l+m+mi}{0}\PY{p}{]}\PY{p}{)}\PY{p}{)}

\PY{c+c1}{\PYZsh{} Squared sum of the difference}
\PY{n+nb}{print}\PY{p}{(}
    \PY{l+s+s2}{\PYZdq{}}\PY{l+s+s2}{Sum of squares: }\PY{l+s+si}{\PYZpc{}0.4f}\PY{l+s+s2}{\PYZdq{}} \PY{o}{\PYZpc{}} \PY{p}{(}\PY{n}{np}\PY{o}{.}\PY{n}{sum}\PY{p}{(}\PY{p}{(}\PY{n}{two\PYZus{}cities}\PY{o}{.}\PY{n}{iloc}\PY{p}{[}\PY{l+m+mi}{1}\PY{p}{]} \PY{o}{\PYZhy{}} \PY{n}{two\PYZus{}cities}\PY{o}{.}\PY{n}{iloc}\PY{p}{[}\PY{l+m+mi}{0}\PY{p}{]}\PY{p}{)} \PY{o}{*}\PY{o}{*} \PY{l+m+mi}{2}\PY{p}{)}\PY{p}{)}
\PY{p}{)}

\PY{c+c1}{\PYZsh{} Take the square root}
\PY{n+nb}{print}\PY{p}{(}
    \PY{l+s+s2}{\PYZdq{}}\PY{l+s+s2}{Euclidean distance between cities: }\PY{l+s+si}{\PYZpc{}0.4f}\PY{l+s+s2}{\PYZdq{}}
    \PY{o}{\PYZpc{}} \PY{p}{(}\PY{n}{np}\PY{o}{.}\PY{n}{sqrt}\PY{p}{(}\PY{n}{np}\PY{o}{.}\PY{n}{sum}\PY{p}{(}\PY{p}{(}\PY{n}{two\PYZus{}cities}\PY{o}{.}\PY{n}{iloc}\PY{p}{[}\PY{l+m+mi}{1}\PY{p}{]} \PY{o}{\PYZhy{}} \PY{n}{two\PYZus{}cities}\PY{o}{.}\PY{n}{iloc}\PY{p}{[}\PY{l+m+mi}{0}\PY{p}{]}\PY{p}{)} \PY{o}{*}\PY{o}{*} \PY{l+m+mi}{2}\PY{p}{)}\PY{p}{)}\PY{p}{)}
\PY{p}{)}
\end{Verbatim}
\end{tcolorbox}

    \begin{Verbatim}[commandchars=\\\{\}]
Subtract the cities:
longitude   -7.2488
latitude    -5.3856
dtype: float64

Sum of squares: 81.5498
Euclidean distance between cities: 9.0305
    \end{Verbatim}

    \begin{tcolorbox}[breakable, size=fbox, boxrule=1pt, pad at break*=1mm,colback=cellbackground, colframe=cellborder]
\prompt{In}{incolor}{14}{\boxspacing}
\begin{Verbatim}[commandchars=\\\{\}]
\PY{n}{two\PYZus{}cities}
\end{Verbatim}
\end{tcolorbox}

            \begin{tcolorbox}[breakable, size=fbox, boxrule=.5pt, pad at break*=1mm, opacityfill=0]
\prompt{Out}{outcolor}{14}{\boxspacing}
\begin{Verbatim}[commandchars=\\\{\}]
    longitude  latitude
69  -104.8253   38.8340
35  -112.0741   33.4484
\end{Verbatim}
\end{tcolorbox}
        
    \begin{tcolorbox}[breakable, size=fbox, boxrule=1pt, pad at break*=1mm,colback=cellbackground, colframe=cellborder]
\prompt{In}{incolor}{15}{\boxspacing}
\begin{Verbatim}[commandchars=\\\{\}]
\PY{c+c1}{\PYZsh{} Euclidean distance using sklearn}
\PY{k+kn}{from} \PY{n+nn}{sklearn}\PY{n+nn}{.}\PY{n+nn}{metrics}\PY{n+nn}{.}\PY{n+nn}{pairwise} \PY{k+kn}{import} \PY{n}{euclidean\PYZus{}distances}

\PY{n}{euclidean\PYZus{}distances}\PY{p}{(}\PY{n}{two\PYZus{}cities}\PY{p}{)}
\end{Verbatim}
\end{tcolorbox}

            \begin{tcolorbox}[breakable, size=fbox, boxrule=.5pt, pad at break*=1mm, opacityfill=0]
\prompt{Out}{outcolor}{15}{\boxspacing}
\begin{Verbatim}[commandchars=\\\{\}]
array([[0.        , 9.03049217],
       [9.03049217, 0.        ]])
\end{Verbatim}
\end{tcolorbox}
        
    Note: \texttt{scikit-learn} supports a number of other
\href{https://scikit-learn.org/stable/modules/generated/sklearn.neighbors.DistanceMetric.html}{distance
metrics}.

    \subsubsection{Finding the nearest
neighbour}\label{finding-the-nearest-neighbour}

\begin{itemize}
\tightlist
\item
  Let's look at distances from all cities to all other cities
\end{itemize}

    \begin{tcolorbox}[breakable, size=fbox, boxrule=1pt, pad at break*=1mm,colback=cellbackground, colframe=cellborder]
\prompt{In}{incolor}{16}{\boxspacing}
\begin{Verbatim}[commandchars=\\\{\}]
\PY{n}{dists} \PY{o}{=} \PY{n}{euclidean\PYZus{}distances}\PY{p}{(}\PY{n}{X\PYZus{}cities}\PY{p}{)}
\PY{n}{np}\PY{o}{.}\PY{n}{fill\PYZus{}diagonal}\PY{p}{(}\PY{n}{dists}\PY{p}{,} \PY{n}{np}\PY{o}{.}\PY{n}{inf}\PY{p}{)}
\PY{n}{dists}\PY{o}{.}\PY{n}{shape}
\end{Verbatim}
\end{tcolorbox}

            \begin{tcolorbox}[breakable, size=fbox, boxrule=.5pt, pad at break*=1mm, opacityfill=0]
\prompt{Out}{outcolor}{16}{\boxspacing}
\begin{Verbatim}[commandchars=\\\{\}]
(209, 209)
\end{Verbatim}
\end{tcolorbox}
        
    \begin{tcolorbox}[breakable, size=fbox, boxrule=1pt, pad at break*=1mm,colback=cellbackground, colframe=cellborder]
\prompt{In}{incolor}{17}{\boxspacing}
\begin{Verbatim}[commandchars=\\\{\}]
\PY{n}{pd}\PY{o}{.}\PY{n}{DataFrame}\PY{p}{(}\PY{n}{dists}\PY{p}{)}
\end{Verbatim}
\end{tcolorbox}

            \begin{tcolorbox}[breakable, size=fbox, boxrule=.5pt, pad at break*=1mm, opacityfill=0]
\prompt{Out}{outcolor}{17}{\boxspacing}
\begin{Verbatim}[commandchars=\\\{\}]
           0          1          2          3          4          5    \textbackslash{}
0          inf   4.955113   9.869531  10.106452  10.449666  19.381676
1     4.955113        inf  14.677579  14.935802  15.305346  24.308448
2     9.869531  14.677579        inf   0.334411   0.808958  11.115406
3    10.106452  14.935802   0.334411        inf   0.474552  10.781004
4    10.449666  15.305346   0.808958   0.474552        inf  10.306500
..         {\ldots}        {\ldots}        {\ldots}        {\ldots}        {\ldots}        {\ldots}
204  58.289799  63.032656  50.483751  50.150395  49.677629  39.405415
205  64.183423  68.887343  56.512897  56.179123  55.705696  45.418031
206  52.426410  57.253724  44.235152  43.904226  43.435186  33.258427
207  58.033459  62.771969  50.249720  49.916254  49.443317  39.167214
208  51.498562  56.252160  43.699224  43.365623  42.892477  32.612755

           6          7          8          9    {\ldots}        199        200  \textbackslash{}
0    28.366626  33.283857  33.572105  36.180388  {\ldots}   9.834455  58.807684
1    33.200978  38.082949  38.359992  40.957919  {\ldots}  14.668787  63.533498
2    20.528403  25.525757  25.873103  28.479109  {\ldots}   0.277381  51.076798
3    20.194002  25.191396  25.538702  28.144750  {\ldots}   0.275352  50.743133
4    19.719500  24.716985  25.064200  27.670344  {\ldots}   0.675814  50.269880
..         {\ldots}        {\ldots}        {\ldots}        {\ldots}  {\ldots}        {\ldots}        {\ldots}
204  30.043890  25.057003  24.746328  22.127878  {\ldots}  50.333340   0.873356
205  36.031385  31.032874  30.709185  28.088948  {\ldots}  56.358333   5.442806
206  24.059863  19.187663  18.932124  16.380495  {\ldots}  44.100248   7.767852
207  29.799983  24.810368  24.497386  21.878183  {\ldots}  50.098326   0.930123
208  23.244592  18.256813  17.946783  15.328953  {\ldots}  43.546610   7.378764

           201        202        203        204        205        206  \textbackslash{}
0    16.925593  56.951696  59.384127  58.289799  64.183423  52.426410
1    21.656349  61.691640  64.045304  63.032656  68.887343  57.253724
2    10.783789  49.169693  51.934205  50.483751  56.512897  44.235152
3    10.480249  48.836189  51.599860  50.150395  56.179123  43.904226
4    10.051472  48.363192  51.125476  49.677629  55.705696  43.435186
..         {\ldots}        {\ldots}        {\ldots}        {\ldots}        {\ldots}        {\ldots}
204  41.380643   1.345136   3.373031        inf   6.102435   6.957987
205  47.259286   7.369875   5.108681   6.102435        inf  12.950733
206  35.637982   5.930561   9.731583   6.957987  12.950733        inf
207  41.121628   1.082749   3.286821   0.316363   6.303916   6.837848
208  34.596810   5.473691   8.568009   6.800190  12.819584   3.322755

           207        208
0    58.033459  51.498562
1    62.771969  56.252160
2    50.249720  43.699224
3    49.916254  43.365623
4    49.443317  42.892477
..         {\ldots}        {\ldots}
204   0.316363   6.800190
205   6.303916  12.819584
206   6.837848   3.322755
207        inf   6.555740
208   6.555740        inf

[209 rows x 209 columns]
\end{Verbatim}
\end{tcolorbox}
        
    Let's look at the distances between City 0 and some other cities.

    \begin{tcolorbox}[breakable, size=fbox, boxrule=1pt, pad at break*=1mm,colback=cellbackground, colframe=cellborder]
\prompt{In}{incolor}{18}{\boxspacing}
\begin{Verbatim}[commandchars=\\\{\}]
\PY{n+nb}{print}\PY{p}{(}\PY{l+s+s2}{\PYZdq{}}\PY{l+s+s2}{Feature vector for city 0: }\PY{l+s+se}{\PYZbs{}n}\PY{l+s+si}{\PYZpc{}s}\PY{l+s+se}{\PYZbs{}n}\PY{l+s+s2}{\PYZdq{}} \PY{o}{\PYZpc{}} \PY{p}{(}\PY{n}{X\PYZus{}cities}\PY{o}{.}\PY{n}{iloc}\PY{p}{[}\PY{l+m+mi}{0}\PY{p}{]}\PY{p}{)}\PY{p}{)}
\PY{n+nb}{print}\PY{p}{(}\PY{l+s+s2}{\PYZdq{}}\PY{l+s+s2}{Distances from city 0 to the first 5 cities: }\PY{l+s+si}{\PYZpc{}s}\PY{l+s+s2}{\PYZdq{}} \PY{o}{\PYZpc{}} \PY{p}{(}\PY{n}{dists}\PY{p}{[}\PY{l+m+mi}{0}\PY{p}{]}\PY{p}{[}\PY{p}{:}\PY{l+m+mi}{5}\PY{p}{]}\PY{p}{)}\PY{p}{)}
\PY{c+c1}{\PYZsh{} We can find the closest city with `np.argmin`:}
\PY{n+nb}{print}\PY{p}{(}
    \PY{l+s+s2}{\PYZdq{}}\PY{l+s+s2}{The closest city from city 0 is: }\PY{l+s+si}{\PYZpc{}d}\PY{l+s+s2}{ }\PY{l+s+se}{\PYZbs{}n}\PY{l+s+se}{\PYZbs{}n}\PY{l+s+s2}{with feature vector: }\PY{l+s+se}{\PYZbs{}n}\PY{l+s+si}{\PYZpc{}s}\PY{l+s+s2}{\PYZdq{}}
    \PY{o}{\PYZpc{}} \PY{p}{(}\PY{n}{np}\PY{o}{.}\PY{n}{argmin}\PY{p}{(}\PY{n}{dists}\PY{p}{[}\PY{l+m+mi}{0}\PY{p}{]}\PY{p}{)}\PY{p}{,} \PY{n}{X\PYZus{}cities}\PY{o}{.}\PY{n}{iloc}\PY{p}{[}\PY{n}{np}\PY{o}{.}\PY{n}{argmin}\PY{p}{(}\PY{n}{dists}\PY{p}{[}\PY{l+m+mi}{0}\PY{p}{]}\PY{p}{)}\PY{p}{]}\PY{p}{)}
\PY{p}{)}
\end{Verbatim}
\end{tcolorbox}

    \begin{Verbatim}[commandchars=\\\{\}]
Feature vector for city 0:
longitude   -130.0437
latitude      55.9773
Name: 0, dtype: float64

Distances from city 0 to the first 5 cities: [        inf  4.95511263  9.869531
10.10645223 10.44966612]
The closest city from city 0 is: 81

with feature vector:
longitude   -129.9912
latitude      55.9383
Name: 81, dtype: float64
    \end{Verbatim}

    Ok, so the closest city to City 0 is City 81.

    \subsubsection{Question}\label{question}

\begin{itemize}
\tightlist
\item
  Why did we set the diagonal entries to infinity before finding the
  closest city?
\end{itemize}

    \subsubsection{Finding the distances to a query
point}\label{finding-the-distances-to-a-query-point}

We can also find the distances to a new ``test'' or ``query'' city:

    \begin{tcolorbox}[breakable, size=fbox, boxrule=1pt, pad at break*=1mm,colback=cellbackground, colframe=cellborder]
\prompt{In}{incolor}{19}{\boxspacing}
\begin{Verbatim}[commandchars=\\\{\}]
\PY{c+c1}{\PYZsh{} Let\PYZsq{}s find a city that\PYZsq{}s closest to the a query city}
\PY{n}{query\PYZus{}point} \PY{o}{=} \PY{p}{[}\PY{p}{[}\PY{o}{\PYZhy{}}\PY{l+m+mi}{80}\PY{p}{,} \PY{l+m+mi}{25}\PY{p}{]}\PY{p}{]}

\PY{n}{dists} \PY{o}{=} \PY{n}{euclidean\PYZus{}distances}\PY{p}{(}\PY{n}{X\PYZus{}cities}\PY{p}{,} \PY{n}{query\PYZus{}point}\PY{p}{)}
\PY{n}{dists}\PY{p}{[}\PY{l+m+mi}{0}\PY{p}{:}\PY{l+m+mi}{10}\PY{p}{]}
\end{Verbatim}
\end{tcolorbox}

            \begin{tcolorbox}[breakable, size=fbox, boxrule=.5pt, pad at break*=1mm, opacityfill=0]
\prompt{Out}{outcolor}{19}{\boxspacing}
\begin{Verbatim}[commandchars=\\\{\}]
array([[58.85545875],
       [63.80062924],
       [49.30530902],
       [49.01473536],
       [48.60495488],
       [39.96834506],
       [32.92852376],
       [29.53520104],
       [29.52881619],
       [27.84679073]])
\end{Verbatim}
\end{tcolorbox}
        
    \begin{tcolorbox}[breakable, size=fbox, boxrule=1pt, pad at break*=1mm,colback=cellbackground, colframe=cellborder]
\prompt{In}{incolor}{20}{\boxspacing}
\begin{Verbatim}[commandchars=\\\{\}]
\PY{c+c1}{\PYZsh{} The query point is closest to}
\PY{n+nb}{print}\PY{p}{(}
    \PY{l+s+s2}{\PYZdq{}}\PY{l+s+s2}{The query point }\PY{l+s+si}{\PYZpc{}s}\PY{l+s+s2}{ is closest to the city with index }\PY{l+s+si}{\PYZpc{}d}\PY{l+s+s2}{ and the distance between them is: }\PY{l+s+si}{\PYZpc{}0.4f}\PY{l+s+s2}{\PYZdq{}}
    \PY{o}{\PYZpc{}} \PY{p}{(}\PY{n}{query\PYZus{}point}\PY{p}{,} \PY{n}{np}\PY{o}{.}\PY{n}{argmin}\PY{p}{(}\PY{n}{dists}\PY{p}{)}\PY{p}{,} \PY{n}{dists}\PY{p}{[}\PY{n}{np}\PY{o}{.}\PY{n}{argmin}\PY{p}{(}\PY{n}{dists}\PY{p}{)}\PY{p}{]}\PY{p}{)}
\PY{p}{)}
\end{Verbatim}
\end{tcolorbox}

    \begin{Verbatim}[commandchars=\\\{\}]
The query point [[-80, 25]] is closest to the city with index 72 and the
distance between them is: 0.7982
    \end{Verbatim}

    

    \subsection{\texorpdfstring{\(k\)-Nearest Neighbours (\(k\)-NNs)
{[}\href{https://youtu.be/bENDqXKJLmg}{video}{]}}{k-Nearest Neighbours (k-NNs) {[}video{]}}}\label{k-nearest-neighbours-k-nns-video}

    \begin{tcolorbox}[breakable, size=fbox, boxrule=1pt, pad at break*=1mm,colback=cellbackground, colframe=cellborder]
\prompt{In}{incolor}{21}{\boxspacing}
\begin{Verbatim}[commandchars=\\\{\}]
\PY{n}{small\PYZus{}cities} \PY{o}{=} \PY{n}{cities\PYZus{}df}\PY{o}{.}\PY{n}{sample}\PY{p}{(}\PY{l+m+mi}{30}\PY{p}{,} \PY{n}{random\PYZus{}state}\PY{o}{=}\PY{l+m+mi}{90}\PY{p}{)}
\PY{n}{one\PYZus{}city} \PY{o}{=} \PY{n}{small\PYZus{}cities}\PY{o}{.}\PY{n}{sample}\PY{p}{(}\PY{l+m+mi}{1}\PY{p}{,} \PY{n}{random\PYZus{}state}\PY{o}{=}\PY{l+m+mi}{44}\PY{p}{)}
\PY{n}{small\PYZus{}train\PYZus{}df} \PY{o}{=} \PY{n}{pd}\PY{o}{.}\PY{n}{concat}\PY{p}{(}\PY{p}{[}\PY{n}{small\PYZus{}cities}\PY{p}{,} \PY{n}{one\PYZus{}city}\PY{p}{]}\PY{p}{)}\PY{o}{.}\PY{n}{drop\PYZus{}duplicates}\PY{p}{(}\PY{n}{keep}\PY{o}{=}\PY{k+kc}{False}\PY{p}{)}
\end{Verbatim}
\end{tcolorbox}

    \begin{tcolorbox}[breakable, size=fbox, boxrule=1pt, pad at break*=1mm,colback=cellbackground, colframe=cellborder]
\prompt{In}{incolor}{22}{\boxspacing}
\begin{Verbatim}[commandchars=\\\{\}]
\PY{n}{X\PYZus{}small\PYZus{}cities} \PY{o}{=} \PY{n}{small\PYZus{}train\PYZus{}df}\PY{o}{.}\PY{n}{drop}\PY{p}{(}\PY{n}{columns}\PY{o}{=}\PY{p}{[}\PY{l+s+s2}{\PYZdq{}}\PY{l+s+s2}{country}\PY{l+s+s2}{\PYZdq{}}\PY{p}{]}\PY{p}{)}\PY{o}{.}\PY{n}{to\PYZus{}numpy}\PY{p}{(}\PY{p}{)}
\PY{n}{y\PYZus{}small\PYZus{}cities} \PY{o}{=} \PY{n}{small\PYZus{}train\PYZus{}df}\PY{p}{[}\PY{l+s+s2}{\PYZdq{}}\PY{l+s+s2}{country}\PY{l+s+s2}{\PYZdq{}}\PY{p}{]}\PY{o}{.}\PY{n}{to\PYZus{}numpy}\PY{p}{(}\PY{p}{)}
\PY{n}{test\PYZus{}point} \PY{o}{=} \PY{n}{one\PYZus{}city}\PY{p}{[}\PY{p}{[}\PY{l+s+s2}{\PYZdq{}}\PY{l+s+s2}{longitude}\PY{l+s+s2}{\PYZdq{}}\PY{p}{,} \PY{l+s+s2}{\PYZdq{}}\PY{l+s+s2}{latitude}\PY{l+s+s2}{\PYZdq{}}\PY{p}{]}\PY{p}{]}\PY{o}{.}\PY{n}{to\PYZus{}numpy}\PY{p}{(}\PY{p}{)}
\end{Verbatim}
\end{tcolorbox}

    \begin{tcolorbox}[breakable, size=fbox, boxrule=1pt, pad at break*=1mm,colback=cellbackground, colframe=cellborder]
\prompt{In}{incolor}{23}{\boxspacing}
\begin{Verbatim}[commandchars=\\\{\}]
\PY{n}{plot\PYZus{}train\PYZus{}test\PYZus{}points}\PY{p}{(}
    \PY{n}{X\PYZus{}small\PYZus{}cities}\PY{p}{,}
    \PY{n}{y\PYZus{}small\PYZus{}cities}\PY{p}{,}
    \PY{n}{test\PYZus{}point}\PY{p}{,}
    \PY{n}{class\PYZus{}names}\PY{o}{=}\PY{p}{[}\PY{l+s+s2}{\PYZdq{}}\PY{l+s+s2}{Canada}\PY{l+s+s2}{\PYZdq{}}\PY{p}{,} \PY{l+s+s2}{\PYZdq{}}\PY{l+s+s2}{USA}\PY{l+s+s2}{\PYZdq{}}\PY{p}{]}\PY{p}{,}
    \PY{n}{test\PYZus{}format}\PY{o}{=}\PY{l+s+s2}{\PYZdq{}}\PY{l+s+s2}{circle}\PY{l+s+s2}{\PYZdq{}}\PY{p}{,}
\PY{p}{)}
\end{Verbatim}
\end{tcolorbox}

    \begin{center}
    \adjustimage{max size={0.9\linewidth}{0.9\paperheight}}{output_55_0.png}
    \end{center}
    { \hspace*{\fill} \\}
    
    \begin{itemize}
\tightlist
\item
  Given a new data point, predict the class of the data point by finding
  the ``closest'' data point in the training set, i.e., by finding its
  ``nearest neighbour'' or majority vote of nearest neighbours.
\end{itemize}

    Suppose we want to predict the class of the black point.\\
- An intuitive way to do this is predict the same label as the
``closest'' point (\(k = 1\)) (1-nearest neighbour) - We would predict a
target of \textbf{USA} in this case.

    \begin{tcolorbox}[breakable, size=fbox, boxrule=1pt, pad at break*=1mm,colback=cellbackground, colframe=cellborder]
\prompt{In}{incolor}{24}{\boxspacing}
\begin{Verbatim}[commandchars=\\\{\}]
\PY{n}{plot\PYZus{}knn\PYZus{}clf}\PY{p}{(}
    \PY{n}{X\PYZus{}small\PYZus{}cities}\PY{p}{,}
    \PY{n}{y\PYZus{}small\PYZus{}cities}\PY{p}{,}
    \PY{n}{test\PYZus{}point}\PY{p}{,}
    \PY{n}{n\PYZus{}neighbors}\PY{o}{=}\PY{l+m+mi}{1}\PY{p}{,}
    \PY{n}{class\PYZus{}names}\PY{o}{=}\PY{p}{[}\PY{l+s+s2}{\PYZdq{}}\PY{l+s+s2}{Canada}\PY{l+s+s2}{\PYZdq{}}\PY{p}{,} \PY{l+s+s2}{\PYZdq{}}\PY{l+s+s2}{USA}\PY{l+s+s2}{\PYZdq{}}\PY{p}{]}\PY{p}{,}
    \PY{n}{test\PYZus{}format}\PY{o}{=}\PY{l+s+s2}{\PYZdq{}}\PY{l+s+s2}{circle}\PY{l+s+s2}{\PYZdq{}}\PY{p}{,}
\PY{p}{)}
\end{Verbatim}
\end{tcolorbox}

    \begin{Verbatim}[commandchars=\\\{\}]
n\_neighbors 1
    \end{Verbatim}

    \begin{center}
    \adjustimage{max size={0.9\linewidth}{0.9\paperheight}}{output_58_1.png}
    \end{center}
    { \hspace*{\fill} \\}
    
    How about using \(k > 1\) to get a more robust estimate? - For example,
we could also use the 3 closest points (\emph{k} = 3) and let them
\textbf{vote} on the correct class.\\
- The \textbf{Canada} class would win in this case.

    \begin{tcolorbox}[breakable, size=fbox, boxrule=1pt, pad at break*=1mm,colback=cellbackground, colframe=cellborder]
\prompt{In}{incolor}{25}{\boxspacing}
\begin{Verbatim}[commandchars=\\\{\}]
\PY{n}{plot\PYZus{}knn\PYZus{}clf}\PY{p}{(}
    \PY{n}{X\PYZus{}small\PYZus{}cities}\PY{p}{,}
    \PY{n}{y\PYZus{}small\PYZus{}cities}\PY{p}{,}
    \PY{n}{test\PYZus{}point}\PY{p}{,}
    \PY{n}{n\PYZus{}neighbors}\PY{o}{=}\PY{l+m+mi}{3}\PY{p}{,}
    \PY{n}{class\PYZus{}names}\PY{o}{=}\PY{p}{[}\PY{l+s+s2}{\PYZdq{}}\PY{l+s+s2}{Canada}\PY{l+s+s2}{\PYZdq{}}\PY{p}{,} \PY{l+s+s2}{\PYZdq{}}\PY{l+s+s2}{USA}\PY{l+s+s2}{\PYZdq{}}\PY{p}{]}\PY{p}{,}
    \PY{n}{test\PYZus{}format}\PY{o}{=}\PY{l+s+s2}{\PYZdq{}}\PY{l+s+s2}{circle}\PY{l+s+s2}{\PYZdq{}}\PY{p}{,}
\PY{p}{)}
\end{Verbatim}
\end{tcolorbox}

    \begin{Verbatim}[commandchars=\\\{\}]
n\_neighbors 3
    \end{Verbatim}

    \begin{center}
    \adjustimage{max size={0.9\linewidth}{0.9\paperheight}}{output_60_1.png}
    \end{center}
    { \hspace*{\fill} \\}
    
    \begin{tcolorbox}[breakable, size=fbox, boxrule=1pt, pad at break*=1mm,colback=cellbackground, colframe=cellborder]
\prompt{In}{incolor}{26}{\boxspacing}
\begin{Verbatim}[commandchars=\\\{\}]
\PY{k+kn}{from} \PY{n+nn}{sklearn}\PY{n+nn}{.}\PY{n+nn}{neighbors} \PY{k+kn}{import} \PY{n}{KNeighborsClassifier}

\PY{n}{k\PYZus{}values} \PY{o}{=} \PY{p}{[}\PY{l+m+mi}{1}\PY{p}{,} \PY{l+m+mi}{3}\PY{p}{]}

\PY{k}{for} \PY{n}{k} \PY{o+ow}{in} \PY{n}{k\PYZus{}values}\PY{p}{:}
    \PY{n}{neigh} \PY{o}{=} \PY{n}{KNeighborsClassifier}\PY{p}{(}\PY{n}{n\PYZus{}neighbors}\PY{o}{=}\PY{n}{k}\PY{p}{)}
    \PY{n}{neigh}\PY{o}{.}\PY{n}{fit}\PY{p}{(}\PY{n}{X\PYZus{}small\PYZus{}cities}\PY{p}{,} \PY{n}{y\PYZus{}small\PYZus{}cities}\PY{p}{)}
    \PY{n+nb}{print}\PY{p}{(}
        \PY{l+s+s2}{\PYZdq{}}\PY{l+s+s2}{Prediction of the black dot with }\PY{l+s+si}{\PYZpc{}d}\PY{l+s+s2}{ neighbours: }\PY{l+s+si}{\PYZpc{}s}\PY{l+s+s2}{\PYZdq{}}
        \PY{o}{\PYZpc{}} \PY{p}{(}\PY{n}{k}\PY{p}{,} \PY{n}{neigh}\PY{o}{.}\PY{n}{predict}\PY{p}{(}\PY{n}{test\PYZus{}point}\PY{p}{)}\PY{p}{)}
    \PY{p}{)}
\end{Verbatim}
\end{tcolorbox}

    \begin{Verbatim}[commandchars=\\\{\}]
Prediction of the black dot with 1 neighbours: ['USA']
Prediction of the black dot with 3 neighbours: ['Canada']
    \end{Verbatim}

    

    \subsubsection{\texorpdfstring{Choosing
\texttt{n\_neighbors}}{Choosing n\_neighbors}}\label{choosing-n_neighbors}

\begin{itemize}
\tightlist
\item
  The primary hyperparameter of the model is \texttt{n\_neighbors}
  (\(k\)) which decides how many neighbours should vote during
  prediction?
\item
  What happens when we play around with \texttt{n\_neighbors}?
\item
  Are we more likely to overfit with a low \texttt{n\_neighbors} or a
  high \texttt{n\_neighbors}?
\item
  Let's examine the effect of the hyperparameter on our cities data.
\end{itemize}

    \begin{tcolorbox}[breakable, size=fbox, boxrule=1pt, pad at break*=1mm,colback=cellbackground, colframe=cellborder]
\prompt{In}{incolor}{27}{\boxspacing}
\begin{Verbatim}[commandchars=\\\{\}]
\PY{n}{X} \PY{o}{=} \PY{n}{cities\PYZus{}df}\PY{o}{.}\PY{n}{drop}\PY{p}{(}\PY{n}{columns}\PY{o}{=}\PY{p}{[}\PY{l+s+s2}{\PYZdq{}}\PY{l+s+s2}{country}\PY{l+s+s2}{\PYZdq{}}\PY{p}{]}\PY{p}{)}
\PY{n}{y} \PY{o}{=} \PY{n}{cities\PYZus{}df}\PY{p}{[}\PY{l+s+s2}{\PYZdq{}}\PY{l+s+s2}{country}\PY{l+s+s2}{\PYZdq{}}\PY{p}{]}

\PY{c+c1}{\PYZsh{} split into train and test sets}
\PY{n}{X\PYZus{}train}\PY{p}{,} \PY{n}{X\PYZus{}test}\PY{p}{,} \PY{n}{y\PYZus{}train}\PY{p}{,} \PY{n}{y\PYZus{}test} \PY{o}{=} \PY{n}{train\PYZus{}test\PYZus{}split}\PY{p}{(}
    \PY{n}{X}\PY{p}{,} \PY{n}{y}\PY{p}{,} \PY{n}{test\PYZus{}size}\PY{o}{=}\PY{l+m+mf}{0.1}\PY{p}{,} \PY{n}{random\PYZus{}state}\PY{o}{=}\PY{l+m+mi}{123}
\PY{p}{)}
\end{Verbatim}
\end{tcolorbox}

    \begin{tcolorbox}[breakable, size=fbox, boxrule=1pt, pad at break*=1mm,colback=cellbackground, colframe=cellborder]
\prompt{In}{incolor}{28}{\boxspacing}
\begin{Verbatim}[commandchars=\\\{\}]
\PY{n}{k} \PY{o}{=} \PY{l+m+mi}{1}
\PY{n}{knn1} \PY{o}{=} \PY{n}{KNeighborsClassifier}\PY{p}{(}\PY{n}{n\PYZus{}neighbors}\PY{o}{=}\PY{n}{k}\PY{p}{)}
\PY{n}{scores} \PY{o}{=} \PY{n}{cross\PYZus{}validate}\PY{p}{(}\PY{n}{knn1}\PY{p}{,} \PY{n}{X\PYZus{}train}\PY{p}{,} \PY{n}{y\PYZus{}train}\PY{p}{,} \PY{n}{return\PYZus{}train\PYZus{}score}\PY{o}{=}\PY{k+kc}{True}\PY{p}{)}
\PY{n}{pd}\PY{o}{.}\PY{n}{DataFrame}\PY{p}{(}\PY{n}{scores}\PY{p}{)}
\end{Verbatim}
\end{tcolorbox}

            \begin{tcolorbox}[breakable, size=fbox, boxrule=.5pt, pad at break*=1mm, opacityfill=0]
\prompt{Out}{outcolor}{28}{\boxspacing}
\begin{Verbatim}[commandchars=\\\{\}]
   fit\_time  score\_time  test\_score  train\_score
0  0.002557    0.004426    0.710526          1.0
1  0.002147    0.001667    0.684211          1.0
2  0.000903    0.001469    0.842105          1.0
3  0.001034    0.001541    0.702703          1.0
4  0.000863    0.001394    0.837838          1.0
\end{Verbatim}
\end{tcolorbox}
        
    \begin{tcolorbox}[breakable, size=fbox, boxrule=1pt, pad at break*=1mm,colback=cellbackground, colframe=cellborder]
\prompt{In}{incolor}{29}{\boxspacing}
\begin{Verbatim}[commandchars=\\\{\}]
\PY{n}{k} \PY{o}{=} \PY{l+m+mi}{100}
\PY{n}{knn100} \PY{o}{=} \PY{n}{KNeighborsClassifier}\PY{p}{(}\PY{n}{n\PYZus{}neighbors}\PY{o}{=}\PY{n}{k}\PY{p}{)}
\PY{n}{scores} \PY{o}{=} \PY{n}{cross\PYZus{}validate}\PY{p}{(}\PY{n}{knn100}\PY{p}{,} \PY{n}{X\PYZus{}train}\PY{p}{,} \PY{n}{y\PYZus{}train}\PY{p}{,} \PY{n}{return\PYZus{}train\PYZus{}score}\PY{o}{=}\PY{k+kc}{True}\PY{p}{)}
\PY{n}{pd}\PY{o}{.}\PY{n}{DataFrame}\PY{p}{(}\PY{n}{scores}\PY{p}{)}
\end{Verbatim}
\end{tcolorbox}

            \begin{tcolorbox}[breakable, size=fbox, boxrule=.5pt, pad at break*=1mm, opacityfill=0]
\prompt{Out}{outcolor}{29}{\boxspacing}
\begin{Verbatim}[commandchars=\\\{\}]
   fit\_time  score\_time  test\_score  train\_score
0  0.002463    0.028968    0.605263     0.600000
1  0.000971    0.002364    0.605263     0.600000
2  0.000927    0.002328    0.605263     0.600000
3  0.000917    0.003322    0.594595     0.602649
4  0.000926    0.001788    0.594595     0.602649
\end{Verbatim}
\end{tcolorbox}
        
    \begin{tcolorbox}[breakable, size=fbox, boxrule=1pt, pad at break*=1mm,colback=cellbackground, colframe=cellborder]
\prompt{In}{incolor}{30}{\boxspacing}
\begin{Verbatim}[commandchars=\\\{\}]
\PY{k}{def} \PY{n+nf}{f}\PY{p}{(}\PY{n}{n\PYZus{}neighbors}\PY{o}{=}\PY{l+m+mi}{1}\PY{p}{)}\PY{p}{:}
    \PY{n}{results} \PY{o}{=} \PY{p}{\PYZob{}}\PY{p}{\PYZcb{}}
    \PY{n}{knn} \PY{o}{=} \PY{n}{KNeighborsClassifier}\PY{p}{(}\PY{n}{n\PYZus{}neighbors}\PY{o}{=}\PY{n}{n\PYZus{}neighbors}\PY{p}{)}
    \PY{n}{scores} \PY{o}{=} \PY{n}{cross\PYZus{}validate}\PY{p}{(}\PY{n}{knn}\PY{p}{,} \PY{n}{X\PYZus{}train}\PY{p}{,} \PY{n}{y\PYZus{}train}\PY{p}{,} \PY{n}{return\PYZus{}train\PYZus{}score}\PY{o}{=}\PY{k+kc}{True}\PY{p}{)}
    \PY{n}{results}\PY{p}{[}\PY{l+s+s2}{\PYZdq{}}\PY{l+s+s2}{n\PYZus{}neighbours}\PY{l+s+s2}{\PYZdq{}}\PY{p}{]} \PY{o}{=} \PY{p}{[}\PY{n}{n\PYZus{}neighbors}\PY{p}{]}
    \PY{n}{results}\PY{p}{[}\PY{l+s+s2}{\PYZdq{}}\PY{l+s+s2}{mean\PYZus{}train\PYZus{}score}\PY{l+s+s2}{\PYZdq{}}\PY{p}{]} \PY{o}{=} \PY{p}{[}\PY{n+nb}{round}\PY{p}{(}\PY{n}{scores}\PY{p}{[}\PY{l+s+s2}{\PYZdq{}}\PY{l+s+s2}{train\PYZus{}score}\PY{l+s+s2}{\PYZdq{}}\PY{p}{]}\PY{o}{.}\PY{n}{mean}\PY{p}{(}\PY{p}{)}\PY{p}{,} \PY{l+m+mi}{3}\PY{p}{)}\PY{p}{]}
    \PY{n}{results}\PY{p}{[}\PY{l+s+s2}{\PYZdq{}}\PY{l+s+s2}{mean\PYZus{}valid\PYZus{}score}\PY{l+s+s2}{\PYZdq{}}\PY{p}{]} \PY{o}{=} \PY{p}{[}\PY{n+nb}{round}\PY{p}{(}\PY{n}{scores}\PY{p}{[}\PY{l+s+s2}{\PYZdq{}}\PY{l+s+s2}{test\PYZus{}score}\PY{l+s+s2}{\PYZdq{}}\PY{p}{]}\PY{o}{.}\PY{n}{mean}\PY{p}{(}\PY{p}{)}\PY{p}{,} \PY{l+m+mi}{3}\PY{p}{)}\PY{p}{]}
    \PY{n+nb}{print}\PY{p}{(}\PY{n}{pd}\PY{o}{.}\PY{n}{DataFrame}\PY{p}{(}\PY{n}{results}\PY{p}{)}\PY{p}{)}


\PY{n}{f}\PY{p}{(}\PY{l+m+mi}{11}\PY{p}{)}
\end{Verbatim}
\end{tcolorbox}

    \begin{Verbatim}[commandchars=\\\{\}]
   n\_neighbours  mean\_train\_score  mean\_valid\_score
0            11             0.819             0.803
    \end{Verbatim}

    \begin{tcolorbox}[breakable, size=fbox, boxrule=1pt, pad at break*=1mm,colback=cellbackground, colframe=cellborder]
\prompt{In}{incolor}{31}{\boxspacing}
\begin{Verbatim}[commandchars=\\\{\}]
\PY{n}{plot\PYZus{}knn\PYZus{}decision\PYZus{}boundaries}\PY{p}{(}\PY{n}{X\PYZus{}train}\PY{p}{,} \PY{n}{y\PYZus{}train}\PY{p}{,} \PY{n}{k\PYZus{}values}\PY{o}{=}\PY{p}{[}\PY{l+m+mi}{1}\PY{p}{,} \PY{l+m+mi}{11}\PY{p}{,} \PY{l+m+mi}{100}\PY{p}{]}\PY{p}{)}
\end{Verbatim}
\end{tcolorbox}

    \begin{center}
    \adjustimage{max size={0.9\linewidth}{0.9\paperheight}}{output_68_0.png}
    \end{center}
    { \hspace*{\fill} \\}
    
    \subsubsection{\texorpdfstring{How to choose
\texttt{n\_neighbors}?}{How to choose n\_neighbors?}}\label{how-to-choose-n_neighbors}

\begin{itemize}
\tightlist
\item
  \texttt{n\_neighbors} is a hyperparameter
\item
  We can use hyperparameter optimization to choose
  \texttt{n\_neighbors}.
\end{itemize}

    \begin{tcolorbox}[breakable, size=fbox, boxrule=1pt, pad at break*=1mm,colback=cellbackground, colframe=cellborder]
\prompt{In}{incolor}{32}{\boxspacing}
\begin{Verbatim}[commandchars=\\\{\}]
\PY{n}{results\PYZus{}dict} \PY{o}{=} \PY{p}{\PYZob{}}
    \PY{l+s+s2}{\PYZdq{}}\PY{l+s+s2}{n\PYZus{}neighbors}\PY{l+s+s2}{\PYZdq{}}\PY{p}{:} \PY{p}{[}\PY{p}{]}\PY{p}{,}
    \PY{l+s+s2}{\PYZdq{}}\PY{l+s+s2}{mean\PYZus{}train\PYZus{}score}\PY{l+s+s2}{\PYZdq{}}\PY{p}{:} \PY{p}{[}\PY{p}{]}\PY{p}{,}
    \PY{l+s+s2}{\PYZdq{}}\PY{l+s+s2}{mean\PYZus{}cv\PYZus{}score}\PY{l+s+s2}{\PYZdq{}}\PY{p}{:} \PY{p}{[}\PY{p}{]}\PY{p}{,}
    \PY{l+s+s2}{\PYZdq{}}\PY{l+s+s2}{std\PYZus{}cv\PYZus{}score}\PY{l+s+s2}{\PYZdq{}}\PY{p}{:} \PY{p}{[}\PY{p}{]}\PY{p}{,}
    \PY{l+s+s2}{\PYZdq{}}\PY{l+s+s2}{std\PYZus{}train\PYZus{}score}\PY{l+s+s2}{\PYZdq{}}\PY{p}{:} \PY{p}{[}\PY{p}{]}\PY{p}{,}
\PY{p}{\PYZcb{}}
\PY{n}{param\PYZus{}grid} \PY{o}{=} \PY{p}{\PYZob{}}\PY{l+s+s2}{\PYZdq{}}\PY{l+s+s2}{n\PYZus{}neighbors}\PY{l+s+s2}{\PYZdq{}}\PY{p}{:} \PY{n}{np}\PY{o}{.}\PY{n}{arange}\PY{p}{(}\PY{l+m+mi}{1}\PY{p}{,} \PY{l+m+mi}{50}\PY{p}{,} \PY{l+m+mi}{5}\PY{p}{)}\PY{p}{\PYZcb{}}

\PY{k}{for} \PY{n}{k} \PY{o+ow}{in} \PY{n}{param\PYZus{}grid}\PY{p}{[}\PY{l+s+s2}{\PYZdq{}}\PY{l+s+s2}{n\PYZus{}neighbors}\PY{l+s+s2}{\PYZdq{}}\PY{p}{]}\PY{p}{:}
    \PY{n}{knn} \PY{o}{=} \PY{n}{KNeighborsClassifier}\PY{p}{(}\PY{n}{n\PYZus{}neighbors}\PY{o}{=}\PY{n}{k}\PY{p}{)}
    \PY{n}{scores} \PY{o}{=} \PY{n}{cross\PYZus{}validate}\PY{p}{(}\PY{n}{knn}\PY{p}{,} \PY{n}{X\PYZus{}train}\PY{p}{,} \PY{n}{y\PYZus{}train}\PY{p}{,} \PY{n}{return\PYZus{}train\PYZus{}score}\PY{o}{=}\PY{k+kc}{True}\PY{p}{)}
    \PY{n}{results\PYZus{}dict}\PY{p}{[}\PY{l+s+s2}{\PYZdq{}}\PY{l+s+s2}{n\PYZus{}neighbors}\PY{l+s+s2}{\PYZdq{}}\PY{p}{]}\PY{o}{.}\PY{n}{append}\PY{p}{(}\PY{n}{k}\PY{p}{)}

    \PY{n}{results\PYZus{}dict}\PY{p}{[}\PY{l+s+s2}{\PYZdq{}}\PY{l+s+s2}{mean\PYZus{}cv\PYZus{}score}\PY{l+s+s2}{\PYZdq{}}\PY{p}{]}\PY{o}{.}\PY{n}{append}\PY{p}{(}\PY{n}{np}\PY{o}{.}\PY{n}{mean}\PY{p}{(}\PY{n}{scores}\PY{p}{[}\PY{l+s+s2}{\PYZdq{}}\PY{l+s+s2}{test\PYZus{}score}\PY{l+s+s2}{\PYZdq{}}\PY{p}{]}\PY{p}{)}\PY{p}{)}
    \PY{n}{results\PYZus{}dict}\PY{p}{[}\PY{l+s+s2}{\PYZdq{}}\PY{l+s+s2}{mean\PYZus{}train\PYZus{}score}\PY{l+s+s2}{\PYZdq{}}\PY{p}{]}\PY{o}{.}\PY{n}{append}\PY{p}{(}\PY{n}{np}\PY{o}{.}\PY{n}{mean}\PY{p}{(}\PY{n}{scores}\PY{p}{[}\PY{l+s+s2}{\PYZdq{}}\PY{l+s+s2}{train\PYZus{}score}\PY{l+s+s2}{\PYZdq{}}\PY{p}{]}\PY{p}{)}\PY{p}{)}
    \PY{n}{results\PYZus{}dict}\PY{p}{[}\PY{l+s+s2}{\PYZdq{}}\PY{l+s+s2}{std\PYZus{}cv\PYZus{}score}\PY{l+s+s2}{\PYZdq{}}\PY{p}{]}\PY{o}{.}\PY{n}{append}\PY{p}{(}\PY{n}{scores}\PY{p}{[}\PY{l+s+s2}{\PYZdq{}}\PY{l+s+s2}{test\PYZus{}score}\PY{l+s+s2}{\PYZdq{}}\PY{p}{]}\PY{o}{.}\PY{n}{std}\PY{p}{(}\PY{p}{)}\PY{p}{)}
    \PY{n}{results\PYZus{}dict}\PY{p}{[}\PY{l+s+s2}{\PYZdq{}}\PY{l+s+s2}{std\PYZus{}train\PYZus{}score}\PY{l+s+s2}{\PYZdq{}}\PY{p}{]}\PY{o}{.}\PY{n}{append}\PY{p}{(}\PY{n}{scores}\PY{p}{[}\PY{l+s+s2}{\PYZdq{}}\PY{l+s+s2}{train\PYZus{}score}\PY{l+s+s2}{\PYZdq{}}\PY{p}{]}\PY{o}{.}\PY{n}{std}\PY{p}{(}\PY{p}{)}\PY{p}{)}

\PY{n}{results\PYZus{}df} \PY{o}{=} \PY{n}{pd}\PY{o}{.}\PY{n}{DataFrame}\PY{p}{(}\PY{n}{results\PYZus{}dict}\PY{p}{)}
\end{Verbatim}
\end{tcolorbox}

    \begin{tcolorbox}[breakable, size=fbox, boxrule=1pt, pad at break*=1mm,colback=cellbackground, colframe=cellborder]
\prompt{In}{incolor}{33}{\boxspacing}
\begin{Verbatim}[commandchars=\\\{\}]
\PY{n}{results\PYZus{}df} \PY{o}{=} \PY{n}{results\PYZus{}df}\PY{o}{.}\PY{n}{set\PYZus{}index}\PY{p}{(}\PY{l+s+s2}{\PYZdq{}}\PY{l+s+s2}{n\PYZus{}neighbors}\PY{l+s+s2}{\PYZdq{}}\PY{p}{)}
\PY{n}{results\PYZus{}df}
\end{Verbatim}
\end{tcolorbox}

            \begin{tcolorbox}[breakable, size=fbox, boxrule=.5pt, pad at break*=1mm, opacityfill=0]
\prompt{Out}{outcolor}{33}{\boxspacing}
\begin{Verbatim}[commandchars=\\\{\}]
             mean\_train\_score  mean\_cv\_score  std\_cv\_score  std\_train\_score
n\_neighbors
1                    1.000000       0.755477      0.069530         0.000000
6                    0.831135       0.792603      0.046020         0.013433
11                   0.819152       0.802987      0.041129         0.011336
16                   0.801863       0.782219      0.074141         0.008735
21                   0.777934       0.766430      0.062792         0.016944
26                   0.755364       0.723613      0.061937         0.025910
31                   0.743391       0.707681      0.057646         0.030408
36                   0.728777       0.707681      0.064452         0.021305
41                   0.706128       0.681223      0.061241         0.018310
46                   0.694155       0.660171      0.093390         0.018178
\end{Verbatim}
\end{tcolorbox}
        
    \begin{tcolorbox}[breakable, size=fbox, boxrule=1pt, pad at break*=1mm,colback=cellbackground, colframe=cellborder]
\prompt{In}{incolor}{34}{\boxspacing}
\begin{Verbatim}[commandchars=\\\{\}]
\PY{n}{results\PYZus{}df}\PY{p}{[}\PY{p}{[}\PY{l+s+s2}{\PYZdq{}}\PY{l+s+s2}{mean\PYZus{}train\PYZus{}score}\PY{l+s+s2}{\PYZdq{}}\PY{p}{,} \PY{l+s+s2}{\PYZdq{}}\PY{l+s+s2}{mean\PYZus{}cv\PYZus{}score}\PY{l+s+s2}{\PYZdq{}}\PY{p}{]}\PY{p}{]}\PY{o}{.}\PY{n}{plot}\PY{p}{(}\PY{p}{)}\PY{p}{;}
\end{Verbatim}
\end{tcolorbox}

    \begin{center}
    \adjustimage{max size={0.9\linewidth}{0.9\paperheight}}{output_72_0.png}
    \end{center}
    { \hspace*{\fill} \\}
    
    \begin{tcolorbox}[breakable, size=fbox, boxrule=1pt, pad at break*=1mm,colback=cellbackground, colframe=cellborder]
\prompt{In}{incolor}{35}{\boxspacing}
\begin{Verbatim}[commandchars=\\\{\}]
\PY{n}{best\PYZus{}n\PYZus{}neighbours} \PY{o}{=} \PY{n}{results\PYZus{}df}\PY{o}{.}\PY{n}{idxmax}\PY{p}{(}\PY{p}{)}\PY{p}{[}\PY{l+s+s2}{\PYZdq{}}\PY{l+s+s2}{mean\PYZus{}cv\PYZus{}score}\PY{l+s+s2}{\PYZdq{}}\PY{p}{]}
\PY{n}{best\PYZus{}n\PYZus{}neighbours}
\end{Verbatim}
\end{tcolorbox}

            \begin{tcolorbox}[breakable, size=fbox, boxrule=.5pt, pad at break*=1mm, opacityfill=0]
\prompt{Out}{outcolor}{35}{\boxspacing}
\begin{Verbatim}[commandchars=\\\{\}]
11
\end{Verbatim}
\end{tcolorbox}
        
    Let's try our best model on test data.

    \begin{tcolorbox}[breakable, size=fbox, boxrule=1pt, pad at break*=1mm,colback=cellbackground, colframe=cellborder]
\prompt{In}{incolor}{36}{\boxspacing}
\begin{Verbatim}[commandchars=\\\{\}]
\PY{n}{knn} \PY{o}{=} \PY{n}{KNeighborsClassifier}\PY{p}{(}\PY{n}{n\PYZus{}neighbors}\PY{o}{=}\PY{n}{best\PYZus{}n\PYZus{}neighbours}\PY{p}{)}
\PY{n}{knn}\PY{o}{.}\PY{n}{fit}\PY{p}{(}\PY{n}{X\PYZus{}train}\PY{p}{,} \PY{n}{y\PYZus{}train}\PY{p}{)}
\PY{n+nb}{print}\PY{p}{(}\PY{l+s+s2}{\PYZdq{}}\PY{l+s+s2}{Test accuracy: }\PY{l+s+si}{\PYZpc{}0.3f}\PY{l+s+s2}{\PYZdq{}} \PY{o}{\PYZpc{}} \PY{p}{(}\PY{n}{knn}\PY{o}{.}\PY{n}{score}\PY{p}{(}\PY{n}{X\PYZus{}test}\PY{p}{,} \PY{n}{y\PYZus{}test}\PY{p}{)}\PY{p}{)}\PY{p}{)}
\end{Verbatim}
\end{tcolorbox}

    \begin{Verbatim}[commandchars=\\\{\}]
Test accuracy: 0.905
    \end{Verbatim}

    Seems like we got lucky with the test set here.

    

    \subsection{❓❓ Questions for you}\label{questions-for-you}

    \subsubsection{(iClicker) Exercise 3.1}\label{iclicker-exercise-3.1}

\textbf{iClicker cloud join link: https://join.iclicker.com/WMSX}

\textbf{Select all of the following statements which are TRUE.}

\begin{itemize}
\tightlist
\item
  \begin{enumerate}
  \def\labelenumi{(\Alph{enumi})}
  \tightlist
  \item
    Analogy-based models find examples from the test set that are most
    similar to the query example we are predicting.
  \end{enumerate}
\item
  \begin{enumerate}
  \def\labelenumi{(\Alph{enumi})}
  \setcounter{enumi}{1}
  \tightlist
  \item
    Euclidean distance will always have a non-negative value.
  \end{enumerate}
\item
  \begin{enumerate}
  \def\labelenumi{(\Alph{enumi})}
  \setcounter{enumi}{2}
  \tightlist
  \item
    With \(k\)-NN, setting the hyperparameter \(k\) to larger values
    typically reduces training error.
  \end{enumerate}
\item
  \begin{enumerate}
  \def\labelenumi{(\Alph{enumi})}
  \setcounter{enumi}{3}
  \tightlist
  \item
    Similar to decision trees, \(k\)-NNs finds a small set of good
    features.
  \end{enumerate}
\item
  \begin{enumerate}
  \def\labelenumi{(\Alph{enumi})}
  \setcounter{enumi}{4}
  \tightlist
  \item
    In \(k\)-NN, with \(k > 1\), the classification of the closest
    neighbour to the test example always contributes the most to the
    prediction.
  \end{enumerate}
\end{itemize}

    \subsection{Break (5 min)}\label{break-5-min}

\includegraphics{/Users/charleenchu/Desktop/school/cpsc330-2023W2/lectures/img/eva-coffee.png}

    

    \subsection{\texorpdfstring{More on \(k\)-NNs
{[}\href{https://youtu.be/IRGbqi5S9gQ}{video}{]}}{More on k-NNs {[}video{]}}}\label{more-on-k-nns-video}

    \subsubsection{\texorpdfstring{Other useful arguments of
\texttt{KNeighborsClassifier}}{Other useful arguments of KNeighborsClassifier}}\label{other-useful-arguments-of-kneighborsclassifier}

\begin{itemize}
\tightlist
\item
  \texttt{weights} \(\rightarrow\) When predicting label, you can assign
  higher weight to the examples which are closer to the query example.\\
\item
  Exercise for you: Play around with this argument. Do you get a better
  validation score?
\end{itemize}

    \subsubsection{\texorpdfstring{Regression with \(k\)-nearest neighbours
(\(k\)-NNs)}{Regression with k-nearest neighbours (k-NNs)}}\label{regression-with-k-nearest-neighbours-k-nns}

\begin{itemize}
\tightlist
\item
  Can we solve regression problems with \(k\)-nearest neighbours
  algorithm?
\item
  In \(k\)-NN regression we take the average of the \(k\)-nearest
  neighbours.
\item
  We can also have weighted regression.
\end{itemize}

See an example of regression in the lecture notes.

    \begin{tcolorbox}[breakable, size=fbox, boxrule=1pt, pad at break*=1mm,colback=cellbackground, colframe=cellborder]
\prompt{In}{incolor}{38}{\boxspacing}
\begin{Verbatim}[commandchars=\\\{\}]
\PY{n}{plot\PYZus{}knn\PYZus{}regression}\PY{p}{(}\PY{n}{n\PYZus{}neighbors}\PY{o}{=}\PY{l+m+mi}{1}\PY{p}{)}
\end{Verbatim}
\end{tcolorbox}

    \begin{center}
    \adjustimage{max size={0.9\linewidth}{0.9\paperheight}}{output_85_0.png}
    \end{center}
    { \hspace*{\fill} \\}
    
    \begin{tcolorbox}[breakable, size=fbox, boxrule=1pt, pad at break*=1mm,colback=cellbackground, colframe=cellborder]
\prompt{In}{incolor}{39}{\boxspacing}
\begin{Verbatim}[commandchars=\\\{\}]
\PY{n}{plot\PYZus{}knn\PYZus{}regression}\PY{p}{(}\PY{n}{n\PYZus{}neighbors}\PY{o}{=}\PY{l+m+mi}{3}\PY{p}{)}
\end{Verbatim}
\end{tcolorbox}

    \begin{center}
    \adjustimage{max size={0.9\linewidth}{0.9\paperheight}}{output_86_0.png}
    \end{center}
    { \hspace*{\fill} \\}
    
    \subsubsection{\texorpdfstring{Pros of \(k\)-NNs for supervised
learning}{Pros of k-NNs for supervised learning}}\label{pros-of-k-nns-for-supervised-learning}

\begin{itemize}
\tightlist
\item
  Easy to understand, interpret.
\item
  Simple hyperparameter \(k\) (\texttt{n\_neighbors}) controlling the
  fundamental tradeoff.
\item
  Can learn very complex functions given enough data.
\item
  Lazy learning: Takes no time to \texttt{fit}
\end{itemize}

    \subsubsection{\texorpdfstring{Cons of \(k\)-NNs for supervised
learning}{Cons of k-NNs for supervised learning}}\label{cons-of-k-nns-for-supervised-learning}

\begin{itemize}
\tightlist
\item
  Can be potentially be VERY slow during prediction time, especially
  when the training set is very large.
\item
  Often not that great test accuracy compared to the modern approaches.
\item
  It does not work well on datasets with many features or where most
  feature values are 0 most of the time (sparse datasets).
\end{itemize}

    \begin{center}\rule{0.5\linewidth}{0.5pt}\end{center}

For regular \(k\)-NN for supervised learning (not with sparse matrices),
you should scale your features. We'll be looking into it soon. \_\_\_

    \subsubsection{(Optional) Parametric vs non
parametric}\label{optional-parametric-vs-non-parametric}

\begin{itemize}
\tightlist
\item
  You might see a lot of definitions of these terms.
\item
  A simple way to think about this is:

  \begin{itemize}
  \tightlist
  \item
    do you need to store at least \(O(n)\) worth of stuff to make
    predictions? If so, it's non-parametric.
  \end{itemize}
\item
  Non-parametric example: \(k\)-NN is a classic example of
  non-parametric models.\\
\item
  Parametric example: decision stump
\item
  If you want to know more about this, find some reading material
  \href{https://www.cs.ubc.ca/~schmidtm/Courses/340-F16/L6.pdf}{here},
  \href{http://mlss.tuebingen.mpg.de/2015/slides/ghahramani/gp-neural-nets15.pdf}{here},
  and
  \href{https://machinelearningmastery.com/parametric-and-nonparametric-machine-learning-algorithms/}{here}.
\item
  By the way, the terms ``parametric'' and ``non-paramteric'' are often
  used differently by statisticians, see
  \href{https://help.xlstat.com/s/article/what-is-the-difference-between-a-parametric-and-a-nonparametric-test?language=en_US}{here}
  for more\ldots{}
\end{itemize}

    \texttt{\{note\}\ \$\textbackslash{}mathcal\{O\}(n)\$\ is\ referred\ to\ as\ big\ \$\textbackslash{}mathcal\{O\}\$\ notation.\ It\ tells\ you\ how\ fast\ an\ algorithm\ is\ or\ how\ much\ storage\ space\ it\ requires.\ For\ example,\ in\ simple\ terms,\ if\ you\ have\ \$n\$\ examples\ and\ you\ need\ to\ store\ them\ all\ you\ can\ say\ that\ the\ algorithm\ requires\ \$\textbackslash{}mathcal\{O\}(n)\$\ worth\ of\ stuff.}

    \subsubsection{Curse of dimensionality}\label{curse-of-dimensionality}

\begin{itemize}
\tightlist
\item
  Affects all learners but especially bad for nearest-neighbour.
\item
  \(k\)-NN usually works well when the number of dimensions \(d\) is
  small but things fall apart quickly as \(d\) goes up.
\item
  If there are many irrelevant attributes, \(k\)-NN is hopelessly
  confused because all of them contribute to finding similarity between
  examples.
\item
  With enough irrelevant attributes the accidental similarity swamps out
  meaningful similarity and \(k\)-NN is no better than random guessing.
\end{itemize}

    \begin{tcolorbox}[breakable, size=fbox, boxrule=1pt, pad at break*=1mm,colback=cellbackground, colframe=cellborder]
\prompt{In}{incolor}{40}{\boxspacing}
\begin{Verbatim}[commandchars=\\\{\}]
\PY{k+kn}{from} \PY{n+nn}{sklearn}\PY{n+nn}{.}\PY{n+nn}{datasets} \PY{k+kn}{import} \PY{n}{make\PYZus{}classification}

\PY{n}{nfeats\PYZus{}accuracy} \PY{o}{=} \PY{p}{\PYZob{}}\PY{l+s+s2}{\PYZdq{}}\PY{l+s+s2}{nfeats}\PY{l+s+s2}{\PYZdq{}}\PY{p}{:} \PY{p}{[}\PY{p}{]}\PY{p}{,} \PY{l+s+s2}{\PYZdq{}}\PY{l+s+s2}{dummy\PYZus{}valid\PYZus{}accuracy}\PY{l+s+s2}{\PYZdq{}}\PY{p}{:} \PY{p}{[}\PY{p}{]}\PY{p}{,} \PY{l+s+s2}{\PYZdq{}}\PY{l+s+s2}{KNN\PYZus{}valid\PYZus{}accuracy}\PY{l+s+s2}{\PYZdq{}}\PY{p}{:} \PY{p}{[}\PY{p}{]}\PY{p}{\PYZcb{}}
\PY{k}{for} \PY{n}{n\PYZus{}feats} \PY{o+ow}{in} \PY{n+nb}{range}\PY{p}{(}\PY{l+m+mi}{4}\PY{p}{,} \PY{l+m+mi}{2000}\PY{p}{,} \PY{l+m+mi}{100}\PY{p}{)}\PY{p}{:}
    \PY{n}{X}\PY{p}{,} \PY{n}{y} \PY{o}{=} \PY{n}{make\PYZus{}classification}\PY{p}{(}\PY{n}{n\PYZus{}samples}\PY{o}{=}\PY{l+m+mi}{2000}\PY{p}{,} \PY{n}{n\PYZus{}features}\PY{o}{=}\PY{n}{n\PYZus{}feats}\PY{p}{,} \PY{n}{n\PYZus{}classes}\PY{o}{=}\PY{l+m+mi}{2}\PY{p}{)}
    \PY{n}{X\PYZus{}train}\PY{p}{,} \PY{n}{X\PYZus{}test}\PY{p}{,} \PY{n}{y\PYZus{}train}\PY{p}{,} \PY{n}{y\PYZus{}test} \PY{o}{=} \PY{n}{train\PYZus{}test\PYZus{}split}\PY{p}{(}
        \PY{n}{X}\PY{p}{,} \PY{n}{y}\PY{p}{,} \PY{n}{test\PYZus{}size}\PY{o}{=}\PY{l+m+mf}{0.2}\PY{p}{,} \PY{n}{random\PYZus{}state}\PY{o}{=}\PY{l+m+mi}{123}
    \PY{p}{)}
    \PY{n}{dummy} \PY{o}{=} \PY{n}{DummyClassifier}\PY{p}{(}\PY{n}{strategy}\PY{o}{=}\PY{l+s+s2}{\PYZdq{}}\PY{l+s+s2}{most\PYZus{}frequent}\PY{l+s+s2}{\PYZdq{}}\PY{p}{)}
    \PY{n}{dummy\PYZus{}scores} \PY{o}{=} \PY{n}{cross\PYZus{}validate}\PY{p}{(}\PY{n}{dummy}\PY{p}{,} \PY{n}{X\PYZus{}train}\PY{p}{,} \PY{n}{y\PYZus{}train}\PY{p}{,} \PY{n}{return\PYZus{}train\PYZus{}score}\PY{o}{=}\PY{k+kc}{True}\PY{p}{)}

    \PY{n}{knn} \PY{o}{=} \PY{n}{KNeighborsClassifier}\PY{p}{(}\PY{p}{)}
    \PY{n}{scores} \PY{o}{=} \PY{n}{cross\PYZus{}validate}\PY{p}{(}\PY{n}{knn}\PY{p}{,} \PY{n}{X\PYZus{}train}\PY{p}{,} \PY{n}{y\PYZus{}train}\PY{p}{,} \PY{n}{return\PYZus{}train\PYZus{}score}\PY{o}{=}\PY{k+kc}{True}\PY{p}{)}
    \PY{n}{nfeats\PYZus{}accuracy}\PY{p}{[}\PY{l+s+s2}{\PYZdq{}}\PY{l+s+s2}{nfeats}\PY{l+s+s2}{\PYZdq{}}\PY{p}{]}\PY{o}{.}\PY{n}{append}\PY{p}{(}\PY{n}{n\PYZus{}feats}\PY{p}{)}
    \PY{n}{nfeats\PYZus{}accuracy}\PY{p}{[}\PY{l+s+s2}{\PYZdq{}}\PY{l+s+s2}{KNN\PYZus{}valid\PYZus{}accuracy}\PY{l+s+s2}{\PYZdq{}}\PY{p}{]}\PY{o}{.}\PY{n}{append}\PY{p}{(}\PY{n}{np}\PY{o}{.}\PY{n}{mean}\PY{p}{(}\PY{n}{scores}\PY{p}{[}\PY{l+s+s2}{\PYZdq{}}\PY{l+s+s2}{test\PYZus{}score}\PY{l+s+s2}{\PYZdq{}}\PY{p}{]}\PY{p}{)}\PY{p}{)}
    \PY{n}{nfeats\PYZus{}accuracy}\PY{p}{[}\PY{l+s+s2}{\PYZdq{}}\PY{l+s+s2}{dummy\PYZus{}valid\PYZus{}accuracy}\PY{l+s+s2}{\PYZdq{}}\PY{p}{]}\PY{o}{.}\PY{n}{append}\PY{p}{(}\PY{n}{np}\PY{o}{.}\PY{n}{mean}\PY{p}{(}\PY{n}{dummy\PYZus{}scores}\PY{p}{[}\PY{l+s+s2}{\PYZdq{}}\PY{l+s+s2}{test\PYZus{}score}\PY{l+s+s2}{\PYZdq{}}\PY{p}{]}\PY{p}{)}\PY{p}{)}
\end{Verbatim}
\end{tcolorbox}

    \begin{tcolorbox}[breakable, size=fbox, boxrule=1pt, pad at break*=1mm,colback=cellbackground, colframe=cellborder]
\prompt{In}{incolor}{41}{\boxspacing}
\begin{Verbatim}[commandchars=\\\{\}]
\PY{n}{pd}\PY{o}{.}\PY{n}{DataFrame}\PY{p}{(}\PY{n}{nfeats\PYZus{}accuracy}\PY{p}{)}
\end{Verbatim}
\end{tcolorbox}

            \begin{tcolorbox}[breakable, size=fbox, boxrule=.5pt, pad at break*=1mm, opacityfill=0]
\prompt{Out}{outcolor}{41}{\boxspacing}
\begin{Verbatim}[commandchars=\\\{\}]
    nfeats  dummy\_valid\_accuracy  KNN\_valid\_accuracy
0        4              0.506250            0.974375
1      104              0.506875            0.745625
2      204              0.499375            0.771250
3      304              0.506250            0.645000
4      404              0.503750            0.695000
5      504              0.501875            0.626250
6      604              0.510000            0.653125
7      704              0.503125            0.611875
8      804              0.505625            0.580625
9      904              0.503125            0.638750
10    1004              0.500625            0.597500
11    1104              0.504375            0.586875
12    1204              0.503125            0.576875
13    1304              0.503125            0.590000
14    1404              0.503750            0.581875
15    1504              0.502500            0.593750
16    1604              0.505625            0.588125
17    1704              0.501250            0.612500
18    1804              0.501250            0.610625
19    1904              0.506250            0.583750
\end{Verbatim}
\end{tcolorbox}
        
    

    \subsection{\texorpdfstring{Support Vector Machines (SVMs) with RBF
kernel
{[}\href{https://youtu.be/ic_zqOhi020}{video}{]}}{Support Vector Machines (SVMs) with RBF kernel {[}video{]}}}\label{support-vector-machines-svms-with-rbf-kernel-video}

\begin{itemize}
\tightlist
\item
  Very high-level overview
\item
  Our goals here are

  \begin{itemize}
  \tightlist
  \item
    Use \texttt{scikit-learn}'s SVM model.
  \item
    Broadly explain the notion of support vectors.\\
  \item
    Broadly explain the similarities and differences between \(k\)-NNs
    and SVM RBFs.
  \item
    Explain how \texttt{C} and \texttt{gamma} hyperparameters control
    the fundamental tradeoff.
  \end{itemize}
\end{itemize}

\begin{quote}
(Optional) RBF stands for radial basis functions. We won't go into what
it means in this video. Refer to
\href{https://www.youtube.com/watch?v=Qc5IyLW_hns}{this video} if you
want to know more.
\end{quote}

    \subsubsection{Overview}\label{overview}

\begin{itemize}
\tightlist
\item
  Another popular similarity-based algorithm is Support Vector Machines
  with RBF Kernel (SVM RBFs)
\item
  Superficially, SVM RBFs are more like weighted \(k\)-NNs.

  \begin{itemize}
  \tightlist
  \item
    The decision boundary is defined by \textbf{a set of positive and
    negative examples} and \textbf{their weights} together with
    \textbf{their similarity measure}.
  \item
    A test example is labeled positive if on average it looks more like
    positive examples than the negative examples.
  \end{itemize}
\end{itemize}

    \begin{itemize}
\tightlist
\item
  The primary difference between \(k\)-NNs and SVM RBFs is that

  \begin{itemize}
  \tightlist
  \item
    Unlike \(k\)-NNs, SVM RBFs only remember the key examples (support
    vectors).
  \item
    SVMs use a different similarity metric which is called a ``kernel''.
    A popular kernel is Radial Basis Functions (RBFs)
  \item
    They usually perform better than \(k\)-NNs!
  \end{itemize}
\end{itemize}

    \subsubsection{Let's explore SVM RBFs}\label{lets-explore-svm-rbfs}

Let's try SVMs on the cities dataset.

    \begin{tcolorbox}[breakable, size=fbox, boxrule=1pt, pad at break*=1mm,colback=cellbackground, colframe=cellborder]
\prompt{In}{incolor}{42}{\boxspacing}
\begin{Verbatim}[commandchars=\\\{\}]
\PY{n}{mglearn\PYZus{}utils}\PY{o}{.}\PY{n}{discrete\PYZus{}scatter}\PY{p}{(}\PY{n}{X\PYZus{}cities}\PY{o}{.}\PY{n}{iloc}\PY{p}{[}\PY{p}{:}\PY{p}{,} \PY{l+m+mi}{0}\PY{p}{]}\PY{p}{,} \PY{n}{X\PYZus{}cities}\PY{o}{.}\PY{n}{iloc}\PY{p}{[}\PY{p}{:}\PY{p}{,} \PY{l+m+mi}{1}\PY{p}{]}\PY{p}{,} \PY{n}{y\PYZus{}cities}\PY{p}{)}
\PY{n}{plt}\PY{o}{.}\PY{n}{xlabel}\PY{p}{(}\PY{l+s+s2}{\PYZdq{}}\PY{l+s+s2}{longitude}\PY{l+s+s2}{\PYZdq{}}\PY{p}{)}
\PY{n}{plt}\PY{o}{.}\PY{n}{ylabel}\PY{p}{(}\PY{l+s+s2}{\PYZdq{}}\PY{l+s+s2}{latitude}\PY{l+s+s2}{\PYZdq{}}\PY{p}{)}
\PY{n}{plt}\PY{o}{.}\PY{n}{legend}\PY{p}{(}\PY{n}{loc}\PY{o}{=}\PY{l+m+mi}{1}\PY{p}{)}\PY{p}{;}
\end{Verbatim}
\end{tcolorbox}

    \begin{center}
    \adjustimage{max size={0.9\linewidth}{0.9\paperheight}}{output_100_0.png}
    \end{center}
    { \hspace*{\fill} \\}
    
    \begin{tcolorbox}[breakable, size=fbox, boxrule=1pt, pad at break*=1mm,colback=cellbackground, colframe=cellborder]
\prompt{In}{incolor}{43}{\boxspacing}
\begin{Verbatim}[commandchars=\\\{\}]
\PY{n}{X\PYZus{}train}\PY{p}{,} \PY{n}{X\PYZus{}test}\PY{p}{,} \PY{n}{y\PYZus{}train}\PY{p}{,} \PY{n}{y\PYZus{}test} \PY{o}{=} \PY{n}{train\PYZus{}test\PYZus{}split}\PY{p}{(}
    \PY{n}{X\PYZus{}cities}\PY{p}{,} \PY{n}{y\PYZus{}cities}\PY{p}{,} \PY{n}{test\PYZus{}size}\PY{o}{=}\PY{l+m+mf}{0.2}\PY{p}{,} \PY{n}{random\PYZus{}state}\PY{o}{=}\PY{l+m+mi}{123}
\PY{p}{)}
\end{Verbatim}
\end{tcolorbox}

    \begin{tcolorbox}[breakable, size=fbox, boxrule=1pt, pad at break*=1mm,colback=cellbackground, colframe=cellborder]
\prompt{In}{incolor}{44}{\boxspacing}
\begin{Verbatim}[commandchars=\\\{\}]
\PY{n}{knn} \PY{o}{=} \PY{n}{KNeighborsClassifier}\PY{p}{(}\PY{n}{n\PYZus{}neighbors}\PY{o}{=}\PY{n}{best\PYZus{}n\PYZus{}neighbours}\PY{p}{)}
\PY{n}{scores} \PY{o}{=} \PY{n}{cross\PYZus{}validate}\PY{p}{(}\PY{n}{knn}\PY{p}{,} \PY{n}{X\PYZus{}train}\PY{p}{,} \PY{n}{y\PYZus{}train}\PY{p}{,} \PY{n}{return\PYZus{}train\PYZus{}score}\PY{o}{=}\PY{k+kc}{True}\PY{p}{)}
\PY{n+nb}{print}\PY{p}{(}\PY{l+s+s2}{\PYZdq{}}\PY{l+s+s2}{Mean validation score }\PY{l+s+si}{\PYZpc{}0.3f}\PY{l+s+s2}{\PYZdq{}} \PY{o}{\PYZpc{}} \PY{p}{(}\PY{n}{np}\PY{o}{.}\PY{n}{mean}\PY{p}{(}\PY{n}{scores}\PY{p}{[}\PY{l+s+s2}{\PYZdq{}}\PY{l+s+s2}{test\PYZus{}score}\PY{l+s+s2}{\PYZdq{}}\PY{p}{]}\PY{p}{)}\PY{p}{)}\PY{p}{)}
\PY{n}{pd}\PY{o}{.}\PY{n}{DataFrame}\PY{p}{(}\PY{n}{scores}\PY{p}{)}
\end{Verbatim}
\end{tcolorbox}

    \begin{Verbatim}[commandchars=\\\{\}]
Mean validation score 0.803
    \end{Verbatim}

            \begin{tcolorbox}[breakable, size=fbox, boxrule=.5pt, pad at break*=1mm, opacityfill=0]
\prompt{Out}{outcolor}{44}{\boxspacing}
\begin{Verbatim}[commandchars=\\\{\}]
   fit\_time  score\_time  test\_score  train\_score
0  0.004504    0.005764    0.794118     0.819549
1  0.001253    0.001671    0.764706     0.819549
2  0.000985    0.001486    0.727273     0.850746
3  0.000888    0.001459    0.787879     0.828358
4  0.000908    0.001487    0.939394     0.783582
\end{Verbatim}
\end{tcolorbox}
        
    \begin{tcolorbox}[breakable, size=fbox, boxrule=1pt, pad at break*=1mm,colback=cellbackground, colframe=cellborder]
\prompt{In}{incolor}{45}{\boxspacing}
\begin{Verbatim}[commandchars=\\\{\}]
\PY{k+kn}{from} \PY{n+nn}{sklearn}\PY{n+nn}{.}\PY{n+nn}{svm} \PY{k+kn}{import} \PY{n}{SVC}

\PY{n}{svm} \PY{o}{=} \PY{n}{SVC}\PY{p}{(}\PY{n}{gamma}\PY{o}{=}\PY{l+m+mf}{0.01}\PY{p}{)}  \PY{c+c1}{\PYZsh{} Ignore gamma for now}
\PY{n}{scores} \PY{o}{=} \PY{n}{cross\PYZus{}validate}\PY{p}{(}\PY{n}{svm}\PY{p}{,} \PY{n}{X\PYZus{}train}\PY{p}{,} \PY{n}{y\PYZus{}train}\PY{p}{,} \PY{n}{return\PYZus{}train\PYZus{}score}\PY{o}{=}\PY{k+kc}{True}\PY{p}{)}
\PY{n+nb}{print}\PY{p}{(}\PY{l+s+s2}{\PYZdq{}}\PY{l+s+s2}{Mean validation score }\PY{l+s+si}{\PYZpc{}0.3f}\PY{l+s+s2}{\PYZdq{}} \PY{o}{\PYZpc{}} \PY{p}{(}\PY{n}{np}\PY{o}{.}\PY{n}{mean}\PY{p}{(}\PY{n}{scores}\PY{p}{[}\PY{l+s+s2}{\PYZdq{}}\PY{l+s+s2}{test\PYZus{}score}\PY{l+s+s2}{\PYZdq{}}\PY{p}{]}\PY{p}{)}\PY{p}{)}\PY{p}{)}
\PY{n}{pd}\PY{o}{.}\PY{n}{DataFrame}\PY{p}{(}\PY{n}{scores}\PY{p}{)}
\end{Verbatim}
\end{tcolorbox}

    \begin{Verbatim}[commandchars=\\\{\}]
Mean validation score 0.820
    \end{Verbatim}

            \begin{tcolorbox}[breakable, size=fbox, boxrule=.5pt, pad at break*=1mm, opacityfill=0]
\prompt{Out}{outcolor}{45}{\boxspacing}
\begin{Verbatim}[commandchars=\\\{\}]
   fit\_time  score\_time  test\_score  train\_score
0  0.007640    0.002785    0.823529     0.842105
1  0.002896    0.001233    0.823529     0.842105
2  0.001268    0.000708    0.727273     0.858209
3  0.001143    0.000745    0.787879     0.843284
4  0.001191    0.000718    0.939394     0.805970
\end{Verbatim}
\end{tcolorbox}
        
    \subsubsection{Decision boundary of
SVMs}\label{decision-boundary-of-svms}

\begin{itemize}
\tightlist
\item
  We can think of SVM with RBF kernel as ``smooth KNN''.
\end{itemize}

    \begin{tcolorbox}[breakable, size=fbox, boxrule=1pt, pad at break*=1mm,colback=cellbackground, colframe=cellborder]
\prompt{In}{incolor}{48}{\boxspacing}
\begin{Verbatim}[commandchars=\\\{\}]
\PY{n}{fig}\PY{p}{,} \PY{n}{axes} \PY{o}{=} \PY{n}{plt}\PY{o}{.}\PY{n}{subplots}\PY{p}{(}\PY{l+m+mi}{1}\PY{p}{,} \PY{l+m+mi}{2}\PY{p}{,} \PY{n}{figsize}\PY{o}{=}\PY{p}{(}\PY{l+m+mi}{16}\PY{p}{,} \PY{l+m+mi}{5}\PY{p}{)}\PY{p}{)}

\PY{k}{for} \PY{n}{clf}\PY{p}{,} \PY{n}{ax} \PY{o+ow}{in} \PY{n+nb}{zip}\PY{p}{(}\PY{p}{[}\PY{n}{knn}\PY{p}{,} \PY{n}{svm}\PY{p}{]}\PY{p}{,} \PY{n}{axes}\PY{p}{)}\PY{p}{:}
    \PY{n}{clf}\PY{o}{.}\PY{n}{fit}\PY{p}{(}\PY{n}{X\PYZus{}train}\PY{p}{,} \PY{n}{y\PYZus{}train}\PY{p}{)}
    \PY{n}{plot\PYZus{}2d\PYZus{}separator}\PY{p}{(}
        \PY{n}{clf}\PY{p}{,} \PY{n}{X\PYZus{}train}\PY{o}{.}\PY{n}{to\PYZus{}numpy}\PY{p}{(}\PY{p}{)}\PY{p}{,} \PY{n}{fill}\PY{o}{=}\PY{k+kc}{True}\PY{p}{,} \PY{n}{eps}\PY{o}{=}\PY{l+m+mf}{0.5}\PY{p}{,} \PY{n}{ax}\PY{o}{=}\PY{n}{ax}\PY{p}{,} \PY{n}{alpha}\PY{o}{=}\PY{l+m+mf}{0.4}
    \PY{p}{)}
    \PY{n}{mglearn\PYZus{}utils}\PY{o}{.}\PY{n}{discrete\PYZus{}scatter}\PY{p}{(}\PY{n}{X\PYZus{}train}\PY{o}{.}\PY{n}{iloc}\PY{p}{[}\PY{p}{:}\PY{p}{,} \PY{l+m+mi}{0}\PY{p}{]}\PY{p}{,} \PY{n}{X\PYZus{}train}\PY{o}{.}\PY{n}{iloc}\PY{p}{[}\PY{p}{:}\PY{p}{,} \PY{l+m+mi}{1}\PY{p}{]}\PY{p}{,} \PY{n}{y\PYZus{}train}\PY{p}{,} \PY{n}{ax}\PY{o}{=}\PY{n}{ax}\PY{p}{)}
    \PY{n}{ax}\PY{o}{.}\PY{n}{set\PYZus{}title}\PY{p}{(}\PY{n}{clf}\PY{p}{)}
    \PY{n}{ax}\PY{o}{.}\PY{n}{set\PYZus{}xlabel}\PY{p}{(}\PY{l+s+s2}{\PYZdq{}}\PY{l+s+s2}{longitude}\PY{l+s+s2}{\PYZdq{}}\PY{p}{)}
    \PY{n}{ax}\PY{o}{.}\PY{n}{set\PYZus{}ylabel}\PY{p}{(}\PY{l+s+s2}{\PYZdq{}}\PY{l+s+s2}{latitude}\PY{l+s+s2}{\PYZdq{}}\PY{p}{)}
\end{Verbatim}
\end{tcolorbox}

    \begin{center}
    \adjustimage{max size={0.9\linewidth}{0.9\paperheight}}{output_105_0.png}
    \end{center}
    { \hspace*{\fill} \\}
    
    \subsubsection{Support vectors}\label{support-vectors}

\begin{itemize}
\item
  Each training example either is or isn't a ``support vector''.

  \begin{itemize}
  \tightlist
  \item
    This gets decided during \texttt{fit}.
  \end{itemize}
\item
  \textbf{Main insight: the decision boundary only depends on the
  support vectors.}
\item
  Let's look at the support vectors.
\end{itemize}

    \begin{tcolorbox}[breakable, size=fbox, boxrule=1pt, pad at break*=1mm,colback=cellbackground, colframe=cellborder]
\prompt{In}{incolor}{49}{\boxspacing}
\begin{Verbatim}[commandchars=\\\{\}]
\PY{k+kn}{from} \PY{n+nn}{sklearn}\PY{n+nn}{.}\PY{n+nn}{datasets} \PY{k+kn}{import} \PY{n}{make\PYZus{}blobs}

\PY{n}{n} \PY{o}{=} \PY{l+m+mi}{20}
\PY{n}{n\PYZus{}classes} \PY{o}{=} \PY{l+m+mi}{2}
\PY{n}{X\PYZus{}toy}\PY{p}{,} \PY{n}{y\PYZus{}toy} \PY{o}{=} \PY{n}{make\PYZus{}blobs}\PY{p}{(}
    \PY{n}{n\PYZus{}samples}\PY{o}{=}\PY{n}{n}\PY{p}{,} \PY{n}{centers}\PY{o}{=}\PY{n}{n\PYZus{}classes}\PY{p}{,} \PY{n}{random\PYZus{}state}\PY{o}{=}\PY{l+m+mi}{300}
\PY{p}{)}  \PY{c+c1}{\PYZsh{} Let\PYZsq{}s generate some fake data}
\end{Verbatim}
\end{tcolorbox}

    \begin{tcolorbox}[breakable, size=fbox, boxrule=1pt, pad at break*=1mm,colback=cellbackground, colframe=cellborder]
\prompt{In}{incolor}{51}{\boxspacing}
\begin{Verbatim}[commandchars=\\\{\}]
\PY{n}{mglearn}\PY{o}{.}\PY{n}{discrete\PYZus{}scatter}\PY{p}{(}\PY{n}{X\PYZus{}toy}\PY{p}{[}\PY{p}{:}\PY{p}{,} \PY{l+m+mi}{0}\PY{p}{]}\PY{p}{,} \PY{n}{X\PYZus{}toy}\PY{p}{[}\PY{p}{:}\PY{p}{,} \PY{l+m+mi}{1}\PY{p}{]}\PY{p}{,} \PY{n}{y\PYZus{}toy}\PY{p}{)}
\PY{n}{plt}\PY{o}{.}\PY{n}{xlabel}\PY{p}{(}\PY{l+s+s2}{\PYZdq{}}\PY{l+s+s2}{Feature 0}\PY{l+s+s2}{\PYZdq{}}\PY{p}{)}
\PY{n}{plt}\PY{o}{.}\PY{n}{ylabel}\PY{p}{(}\PY{l+s+s2}{\PYZdq{}}\PY{l+s+s2}{Feature 1}\PY{l+s+s2}{\PYZdq{}}\PY{p}{)}
\PY{n}{svm} \PY{o}{=} \PY{n}{SVC}\PY{p}{(}\PY{n}{kernel}\PY{o}{=}\PY{l+s+s2}{\PYZdq{}}\PY{l+s+s2}{rbf}\PY{l+s+s2}{\PYZdq{}}\PY{p}{,} \PY{n}{C}\PY{o}{=}\PY{l+m+mi}{10}\PY{p}{,} \PY{n}{gamma}\PY{o}{=}\PY{l+m+mf}{0.1}\PY{p}{)}\PY{o}{.}\PY{n}{fit}\PY{p}{(}\PY{n}{X\PYZus{}toy}\PY{p}{,} \PY{n}{y\PYZus{}toy}\PY{p}{)}
\PY{n}{plot\PYZus{}2d\PYZus{}separator}\PY{p}{(}\PY{n}{svm}\PY{p}{,} \PY{n}{X\PYZus{}toy}\PY{p}{,} \PY{n}{fill}\PY{o}{=}\PY{k+kc}{True}\PY{p}{,} \PY{n}{eps}\PY{o}{=}\PY{l+m+mf}{0.5}\PY{p}{,} \PY{n}{alpha}\PY{o}{=}\PY{l+m+mf}{0.4}\PY{p}{)}
\end{Verbatim}
\end{tcolorbox}

    \begin{center}
    \adjustimage{max size={0.9\linewidth}{0.9\paperheight}}{output_108_0.png}
    \end{center}
    { \hspace*{\fill} \\}
    
    \begin{tcolorbox}[breakable, size=fbox, boxrule=1pt, pad at break*=1mm,colback=cellbackground, colframe=cellborder]
\prompt{In}{incolor}{52}{\boxspacing}
\begin{Verbatim}[commandchars=\\\{\}]
\PY{n}{svm}\PY{o}{.}\PY{n}{support\PYZus{}}
\end{Verbatim}
\end{tcolorbox}

            \begin{tcolorbox}[breakable, size=fbox, boxrule=.5pt, pad at break*=1mm, opacityfill=0]
\prompt{Out}{outcolor}{52}{\boxspacing}
\begin{Verbatim}[commandchars=\\\{\}]
array([ 3,  8,  9, 14, 19,  1,  4,  6, 17], dtype=int32)
\end{Verbatim}
\end{tcolorbox}
        
    \begin{tcolorbox}[breakable, size=fbox, boxrule=1pt, pad at break*=1mm,colback=cellbackground, colframe=cellborder]
\prompt{In}{incolor}{53}{\boxspacing}
\begin{Verbatim}[commandchars=\\\{\}]
\PY{n}{plot\PYZus{}support\PYZus{}vectors}\PY{p}{(}\PY{n}{svm}\PY{p}{,} \PY{n}{X\PYZus{}toy}\PY{p}{,} \PY{n}{y\PYZus{}toy}\PY{p}{)}
\end{Verbatim}
\end{tcolorbox}

    \begin{center}
    \adjustimage{max size={0.9\linewidth}{0.9\paperheight}}{output_110_0.png}
    \end{center}
    { \hspace*{\fill} \\}
    
    The support vectors are the bigger points in the plot above.

    \subsubsection{Hyperparameters of SVM}\label{hyperparameters-of-svm}

\begin{itemize}
\tightlist
\item
  Key hyperparameters of \texttt{rbf} SVM are

  \begin{itemize}
  \tightlist
  \item
    \texttt{gamma}
  \item
    \texttt{C}
  \end{itemize}
\item
  We are not equipped to understand the meaning of these parameters at
  this point but you are expected to describe their relation to the
  fundamental tradeoff.
\end{itemize}

See
\href{https://scikit-learn.org/stable/auto_examples/svm/plot_rbf_parameters.html}{\texttt{scikit-learn}'s
explanation of RBF SVM parameters}.

    \subsubsection{\texorpdfstring{Relation of \texttt{gamma} and the
fundamental
trade-off}{Relation of gamma and the fundamental trade-off}}\label{relation-of-gamma-and-the-fundamental-trade-off}

\begin{itemize}
\tightlist
\item
  \texttt{gamma} controls the complexity (fundamental trade-off), just
  like other hyperparameters we've seen.

  \begin{itemize}
  \tightlist
  \item
    larger \texttt{gamma} \(\rightarrow\) more complex
  \item
    smaller \texttt{gamma} \(\rightarrow\) less complex
  \end{itemize}
\end{itemize}

    \begin{tcolorbox}[breakable, size=fbox, boxrule=1pt, pad at break*=1mm,colback=cellbackground, colframe=cellborder]
\prompt{In}{incolor}{54}{\boxspacing}
\begin{Verbatim}[commandchars=\\\{\}]
\PY{n}{gamma} \PY{o}{=} \PY{p}{[}\PY{l+m+mf}{0.001}\PY{p}{,} \PY{l+m+mf}{0.01}\PY{p}{,} \PY{l+m+mf}{0.1}\PY{p}{,} \PY{l+m+mf}{1.0}\PY{p}{,} \PY{l+m+mf}{10.0}\PY{p}{]}
\PY{n}{plot\PYZus{}svc\PYZus{}gamma}\PY{p}{(}
    \PY{n}{gamma}\PY{p}{,}
    \PY{n}{X\PYZus{}train}\PY{o}{.}\PY{n}{to\PYZus{}numpy}\PY{p}{(}\PY{p}{)}\PY{p}{,}
    \PY{n}{y\PYZus{}train}\PY{o}{.}\PY{n}{to\PYZus{}numpy}\PY{p}{(}\PY{p}{)}\PY{p}{,}
    \PY{n}{x\PYZus{}label}\PY{o}{=}\PY{l+s+s2}{\PYZdq{}}\PY{l+s+s2}{longitude}\PY{l+s+s2}{\PYZdq{}}\PY{p}{,}
    \PY{n}{y\PYZus{}label}\PY{o}{=}\PY{l+s+s2}{\PYZdq{}}\PY{l+s+s2}{latitude}\PY{l+s+s2}{\PYZdq{}}\PY{p}{,}
\PY{p}{)}
\end{Verbatim}
\end{tcolorbox}

    \begin{center}
    \adjustimage{max size={0.9\linewidth}{0.9\paperheight}}{output_114_0.png}
    \end{center}
    { \hspace*{\fill} \\}
    
    \subsubsection{\texorpdfstring{Relation of \texttt{C} and the
fundamental
trade-off}{Relation of C and the fundamental trade-off}}\label{relation-of-c-and-the-fundamental-trade-off}

\begin{itemize}
\tightlist
\item
  \texttt{C} \emph{also} affects the fundamental tradeoff

  \begin{itemize}
  \tightlist
  \item
    larger \texttt{C} \(\rightarrow\) more complex
  \item
    smaller \texttt{C} \(\rightarrow\) less complex
  \end{itemize}
\end{itemize}

    \begin{tcolorbox}[breakable, size=fbox, boxrule=1pt, pad at break*=1mm,colback=cellbackground, colframe=cellborder]
\prompt{In}{incolor}{55}{\boxspacing}
\begin{Verbatim}[commandchars=\\\{\}]
\PY{n}{C} \PY{o}{=} \PY{p}{[}\PY{l+m+mf}{0.1}\PY{p}{,} \PY{l+m+mf}{1.0}\PY{p}{,} \PY{l+m+mf}{100.0}\PY{p}{,} \PY{l+m+mf}{1000.0}\PY{p}{,} \PY{l+m+mf}{100000.0}\PY{p}{]}
\PY{n}{plot\PYZus{}svc\PYZus{}C}\PY{p}{(}
    \PY{n}{C}\PY{p}{,} \PY{n}{X\PYZus{}train}\PY{o}{.}\PY{n}{to\PYZus{}numpy}\PY{p}{(}\PY{p}{)}\PY{p}{,} \PY{n}{y\PYZus{}train}\PY{o}{.}\PY{n}{to\PYZus{}numpy}\PY{p}{(}\PY{p}{)}\PY{p}{,} \PY{n}{x\PYZus{}label}\PY{o}{=}\PY{l+s+s2}{\PYZdq{}}\PY{l+s+s2}{longitude}\PY{l+s+s2}{\PYZdq{}}\PY{p}{,} \PY{n}{y\PYZus{}label}\PY{o}{=}\PY{l+s+s2}{\PYZdq{}}\PY{l+s+s2}{latitude}\PY{l+s+s2}{\PYZdq{}}
\PY{p}{)}
\end{Verbatim}
\end{tcolorbox}

    \begin{center}
    \adjustimage{max size={0.9\linewidth}{0.9\paperheight}}{output_116_0.png}
    \end{center}
    { \hspace*{\fill} \\}
    
    \subsubsection{Search over multiple
hyperparameters}\label{search-over-multiple-hyperparameters}

\begin{itemize}
\tightlist
\item
  So far you have seen how to carry out search over a hyperparameter
\item
  In the above case the best training error is achieved by the most
  complex model (large \texttt{gamma}, large \texttt{C}).
\item
  Best validation error requires a hyperparameter search to balance the
  fundamental tradeoff.

  \begin{itemize}
  \tightlist
  \item
    In general we can't search them one at a time.
  \item
    More on this next week. But if you cannot wait till then, you may
    look up the following:

    \begin{itemize}
    \tightlist
    \item
      \href{https://scikit-learn.org/stable/modules/generated/sklearn.model_selection.GridSearchCV.html}{sklearn.model\_selection.GridSearchCV}
    \item
      \href{https://scikit-learn.org/stable/modules/generated/sklearn.model_selection.RandomizedSearchCV.html}{sklearn.model\_selection.RandomizedSearchCV}
    \end{itemize}
  \end{itemize}
\end{itemize}

    \subsubsection{SVM Regressor}\label{svm-regressor}

\begin{itemize}
\tightlist
\item
  Similar to KNNs, you can use SVMs for regression problems as well.
\item
  See
  \href{https://scikit-learn.org/stable/modules/generated/sklearn.svm.SVR.html}{\texttt{sklearn.svm.SVR}}
  for more details.
\end{itemize}

    \subsection{❓❓ Questions for you}\label{questions-for-you}

    \subsection{(iClicker) Exercise 3.2}\label{iclicker-exercise-3.2}

\textbf{iClicker cloud join link: https://join.iclicker.com/WMSX}

\textbf{Select all of the following statements which are TRUE.}

\begin{itemize}
\tightlist
\item
  \begin{enumerate}
  \def\labelenumi{(\Alph{enumi})}
  \tightlist
  \item
    \(k\)-NN may perform poorly in high-dimensional space (say, \emph{d}
    \textgreater{} 1000).
  \end{enumerate}
\item
  \begin{enumerate}
  \def\labelenumi{(\Alph{enumi})}
  \setcounter{enumi}{1}
  \tightlist
  \item
    In SVM RBF, removing a non-support vector would not change the
    decision boundary.
  \end{enumerate}
\item
  \begin{enumerate}
  \def\labelenumi{(\Alph{enumi})}
  \setcounter{enumi}{2}
  \tightlist
  \item
    In sklearn's SVC classifier, large values of gamma tend to result in
    higher training score but probably lower validation score.
  \end{enumerate}
\item
  \begin{enumerate}
  \def\labelenumi{(\Alph{enumi})}
  \setcounter{enumi}{3}
  \tightlist
  \item
    If we increase both gamma and C, we can't be certain if the model
    becomes more complex or less complex.
  \end{enumerate}
\end{itemize}

    \paragraph{More practice questions}\label{more-practice-questions}

\begin{itemize}
\tightlist
\item
  Check out some more practice questions
  \href{https://ml-learn.mds.ubc.ca/en/module4}{here}.
\end{itemize}

    \subsection{Summary}\label{summary}

\begin{itemize}
\tightlist
\item
  We have KNNs and SVMs as new supervised learning techniques in our
  toolbox.
\item
  These are analogy-based learners and the idea is to assign nearby
  points the same label.
\item
  Unlike decision trees, all features are equally important.
\item
  Both can be used for classification or regression (much like the other
  methods we've seen).
\end{itemize}

    \subsubsection{Coming up:}\label{coming-up}

Lingering questions: - Are we ready to do machine learning on real-world
datasets? - What would happen if we use \(k\)-NNs or SVM RBFs on the
spotify dataset from hw1?\\
- What happens if we have missing values in our data? - What do we do if
we have features with categories or string values?

    \includegraphics{/Users/charleenchu/Desktop/school/cpsc330-2023W2/lectures/img/eva-seeyou.png}


    % Add a bibliography block to the postdoc
    
    
    
\end{document}
